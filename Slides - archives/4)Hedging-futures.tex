\documentclass[table,xcolor=pdftex,dvipsnames]{beamer}\usepackage[]{graphicx}\usepackage[]{color}
%% maxwidth is the original width if it is less than linewidth
%% otherwise use linewidth (to make sure the graphics do not exceed the margin)
\makeatletter
\def\maxwidth{ %
  \ifdim\Gin@nat@width>\linewidth
    \linewidth
  \else
    \Gin@nat@width
  \fi
}
\makeatother

\definecolor{fgcolor}{rgb}{0.345, 0.345, 0.345}
\newcommand{\hlnum}[1]{\textcolor[rgb]{0.686,0.059,0.569}{#1}}%
\newcommand{\hlstr}[1]{\textcolor[rgb]{0.192,0.494,0.8}{#1}}%
\newcommand{\hlcom}[1]{\textcolor[rgb]{0.678,0.584,0.686}{\textit{#1}}}%
\newcommand{\hlopt}[1]{\textcolor[rgb]{0,0,0}{#1}}%
\newcommand{\hlstd}[1]{\textcolor[rgb]{0.345,0.345,0.345}{#1}}%
\newcommand{\hlkwa}[1]{\textcolor[rgb]{0.161,0.373,0.58}{\textbf{#1}}}%
\newcommand{\hlkwb}[1]{\textcolor[rgb]{0.69,0.353,0.396}{#1}}%
\newcommand{\hlkwc}[1]{\textcolor[rgb]{0.333,0.667,0.333}{#1}}%
\newcommand{\hlkwd}[1]{\textcolor[rgb]{0.737,0.353,0.396}{\textbf{#1}}}%
\let\hlipl\hlkwb

\usepackage{framed}
\makeatletter
\newenvironment{kframe}{%
 \def\at@end@of@kframe{}%
 \ifinner\ifhmode%
  \def\at@end@of@kframe{\end{minipage}}%
  \begin{minipage}{\columnwidth}%
 \fi\fi%
 \def\FrameCommand##1{\hskip\@totalleftmargin \hskip-\fboxsep
 \colorbox{shadecolor}{##1}\hskip-\fboxsep
     % There is no \\@totalrightmargin, so:
     \hskip-\linewidth \hskip-\@totalleftmargin \hskip\columnwidth}%
 \MakeFramed {\advance\hsize-\width
   \@totalleftmargin\z@ \linewidth\hsize
   \@setminipage}}%
 {\par\unskip\endMakeFramed%
 \at@end@of@kframe}
\makeatother

\definecolor{shadecolor}{rgb}{.97, .97, .97}
\definecolor{messagecolor}{rgb}{0, 0, 0}
\definecolor{warningcolor}{rgb}{1, 0, 1}
\definecolor{errorcolor}{rgb}{1, 0, 0}
\newenvironment{knitrout}{}{} % an empty environment to be redefined in TeX

\usepackage{alltt}
%\documentclass[table,xcolor=pdftex,dvipsnames, handout]{beamer}

%\usepackage{handoutWithNotes}
%\pgfpagesuselayout{4 on 1 with notes}[letterpaper,border shrink=5mm]

\usepackage{beamerthemesplit}
\usepackage[english]{babel}
\usepackage{amsmath}
\usepackage{amssymb}
\usepackage{amsthm}
\usepackage{verbatim}
\usepackage{graphpap}
\usepackage{epic}
\usepackage{pict2e} %To draw line with any slope
\usepackage{color}
\usepackage{natbib}
\usepackage{enumitem}
\usepackage{booktabs}
\usepackage{xcolor}
\usepackage{textcomp}
%\usepackage{movie15}

\bibliographystyle{ajae}

\newcommand{\p}{\partial}

\newcommand {\framedgraphic}[1] {
        \begin{center}
            \includegraphics[width=\textwidth,height=0.8\textheight,keepaspectratio]{#1}
        \end{center}
        \vspace{-1\baselineskip}
}

\usetheme{Boadilla}
\useoutertheme{shadow}
\usecolortheme{beaver}%seagull
\everymath{\color{blue}}
\everydisplay{\color{blue}}

\usefonttheme{professionalfonts}

\usepackage{hyperref}
\hypersetup{
   colorlinks = {true},
   urlcolor = {blue},
   linkcolor = {black},
   citecolor = {black},
   pdfborderstyle={/S/U/W 1},
   urlbordercolor = 0 0 1,
   citebordercolor = 1 1 1,
   filebordercolor = 1 1 1,
   linkbordercolor = 1 1 1,
   pdfauthor = {Sebastien Pouliot},
}

\widowpenalty=10000 % Avoid single line at the end of a page
\clubpenalty=10000  % Avoid single line at the bottom

\title[Hedging with futures]{Hedging with futures}
\author[Pouliot]{S\'{e}bastien Pouliot}
\institute{Iowa State University}
\date{Fall 2017}
\IfFileExists{upquote.sty}{\usepackage{upquote}}{}
\begin{document}

%%%%%%%%%%%%%%%%%%%%%%%%%%%%%%%%%%%%%%%%%%%%%%%%%%%%%%%%%%%%%%%%%%%%%%%%%%%%%%%%%%

\begin{frame}
\titlepage
\vspace{-0.4in}
\begin{center}
Lecture notes for Econ 235\\
\end{center}
\end{frame}

%%%%%%%%%%%%%%%%%%%%%%%%%%%%%%%%%%%%%%%%%%%%%%%%%%%%%%%%%%%%%%%%%%%%%%%%%%%%%%%%%%
\section{Introduction}

\begin{frame}{Definitions}
\begin{enumerate}[label=\textbullet]
  \item Hedging is the action of taking a position opposite to an existing one to counterbalance gains or losses.
    \begin{enumerate}[label=-]
         \item For example, if long in the cash market, then take a short position in the futures market.
         \item In the opposite case, if short in the cash market, then take a long position in the futures market.
    \end{enumerate}
  \item Simple hedging strategies use the fact that prices in the cash and futures markets move in the same directions. More advanced hedging strategies may use price movements in opposite directions in the cash and futures markets or position in other markets.
  \item We will consider hedging with futures in this section.
\end{enumerate}
\end{frame}

%%%%%%%%%%%%%%%%%%%%%%%%%%%%%%%%%%%%%%%%%%%%%%%%%%%%%%%%%%%%%%%%%%%%%%%%%%%%%%%%%%

\begin{frame}{To read!}
\begin{enumerate}[label=\textbullet]
  \item \href{http://www.cmegroup.com/trading/agricultural/self-study-guide-to-hedging-with-grain-and-oilseed-futures-and-options.html}{Self-Study Guide to Hedging with Grain and Oilseed Futures and Options.}
    \item \href{http://www.cmegroup.com/trading/agricultural/self-study-guide-hedging-livestock-futures-options.html}{Self-Study Guide to Hedging with Livestock Futures and Options.}
  \item McNew's Marketing Guide available on Blackboard.
  \item \href{http://www.extension.iastate.edu/agdm/crops/html/a2-60.html}{Grain Price Hedging Basics from ISU extension.}
\end{enumerate}
\end{frame}

%%%%%%%%%%%%%%%%%%%%%%%%%%%%%%%%%%%%%%%%%%%%%%%%%%%%%%%%%%%%%%%%%%%%%%%%%%%%%%%%%%

\begin{frame}{Why hedge?}
\begin{enumerate}[label=\textbullet]
  \item Arbitrage:
    \begin{enumerate}[label=-]
         \item Can earn a risk-free return by taking advantage of predictable changes in the basis;
         \item One example is a firm that can store a commodity at a lower cost than the change in the basis (accounting for transaction costs).
    \end{enumerate}
  \item Operational hedging:
      \begin{enumerate}[label=-]
            \item Provides flexibility in day-to-day operations and reduces price risk;
            \item Can be used when forward contracting with flexible exchange rate;
            \item Amounts to speculating on the basis.
      \end{enumerate}
  \item Anticipation of transaction:
      \begin{enumerate}[label=-]
            \item Hedging to reduce price risk in anticipation of a forthcoming transaction in the cash market.
      \end{enumerate}
\end{enumerate}
\end{frame}

%%%%%%%%%%%%%%%%%%%%%%%%%%%%%%%%%%%%%%%%%%%%%%%%%%%%%%%%%%%%%%%%%%%%%%%%%%%%%%%%%%

\begin{frame}{Do farmers hedge?}
\begin{enumerate}[label=\textbullet]
\item From \cite{carter2003}:  
  \begin{enumerate}[label=-]
    \item A small fraction of farmers hedge to protect against price risk.
    \item Farmers tend to speculate more than hedge.
  \end{enumerate}
  \item Why so few farmers hedge? 
      \begin{enumerate}[label=-]
            \item Government programs;
            \item Production risk;
            \item Lack of knowledge;
            \item Margin calls make it too risky;
            \item Availability of forward contracts;
            \item Production size does not match size of futures contract;
            \item Can manage price risk with forward contract.
      \end{enumerate}
  \item With less government support, higher commodity prices and greater price volatility, farmers would have more incentives to hedge.
  \item \cite{carter2003} is a fairly old book and it seems that more farmers hedge now.
\end{enumerate}
\end{frame}

%%%%%%%%%%%%%%%%%%%%%%%%%%%%%%%%%%%%%%%%%%%%%%%%%%%%%%%%%%%%%%%%%%%%%%%%%%%%%%%%%%
\section{Hedging strategy}

\begin{frame}{Basic hedging strategy with futures and options}
\begin{enumerate}[label=\textbullet]
  \item We will begin by covering the basics of hedging with futures.
  \item We will focus on hedging to reduce risk from price variability.
\end{enumerate}
\end{frame}

%%%%%%%%%%%%%%%%%%%%%%%%%%%%%%%%%%%%%%%%%%%%%%%%%%%%%%%%%%%%%%%%%%%%%%%%%%%%%%%%%%

\subsection{Short hedging with futures}

\begin{frame}{Short hedging}
\begin{enumerate}[label=\textbullet]
  \item Suppose that you are a corn farmer.
  \item You wish to hedge to protect against price risk.
  \item To simplify, assume for now that the basis is zero.
  \item Suppose that the current price of corn on the December futures market is \$6.00 per bushel.
  \item In hedging, your strategy is to take a position opposite to the one you have on the cash market.
  \item As you are \textbf{long in the cash market}, you must take a \textbf{short position in the futures market}.
\end{enumerate}
\end{frame}

%%%%%%%%%%%%%%%%%%%%%%%%%%%%%%%%%%%%%%%%%%%%%%%%%%%%%%%%%%%%%%%%%%%%%%%%%%%%%%%%%%
\begin{frame}{Example: short hedge}
\begin{enumerate}[label=\textbullet]
  \item Let's consider two cases:
      \begin{enumerate}[label=\arabic*)]
            \item At the time of delivery in December, the price of corn has increased to \$7.00 per bushel.
            \item At the time of delivery in December, the price of corn has declined to \$5.00 per bushel.
      \end{enumerate}
  \item If the price of corn increases to \$7.00 per bushel, then you:
      \begin{enumerate}[label=-]
            \item Gain \$1 per bushel in the cash market;
            \item Lose \$1 per bushel in the futures market;
            \item In total, you do not gain or lose from your hedging position.
      \end{enumerate}
  \item If the price of corn declines to \$5.00 per bushel, then you:
      \begin{enumerate}[label=-]
            \item Lose \$1 per bushel in the cash market;
            \item Gain \$1 per bushel in the futures market;
            \item In total, you do not gain or lose from your hedging position.
      \end{enumerate}
\end{enumerate}
\end{frame}

%%%%%%%%%%%%%%%%%%%%%%%%%%%%%%%%%%%%%%%%%%%%%%%%%%%%%%%%%%%%%%%%%%%%%%%%%%%%%%%%%%

\begin{frame}{Example: short hedge}
\begin{table}
\caption{Price of corn increases}
\scriptsize
\begin{tabular}{l c c}
  \toprule
   & Futures & Cash\\
  \midrule
  May price & \$6.00/bu & \$6.00/bu \\
  \addlinespace[0.075in]
  December price & \$7.00/bu & \$7.00/bu \\
  \cmidrule(r){2-3}
  Gain/loss & -\$1.00/bu & \$1.00/bu \\
  \midrule
  Price of corn at beginning of hedge & & \textbf{\$6.00/bu} \\
  \addlinespace[0.075in]
  Gain/loss from cash position & & \textbf{\$1.00/bu}\\
  \addlinespace[0.075in]
  Gain/loss from futures position & & \textbf{-\$1.00/bu}\\
  \cmidrule(r){3-3}
  Net selling price & & \textbf{\$6.00/bu}\\
  \bottomrule
\end{tabular}
\end{table}
\end{frame}

%%%%%%%%%%%%%%%%%%%%%%%%%%%%%%%%%%%%%%%%%%%%%%%%%%%%%%%%%%%%%%%%%%%%%%%%%%%%%%%%%%

\begin{frame}{Example: short hedge}
\begin{table}
\caption{Price of corn declines}
\scriptsize
\begin{tabular}{l c c}
  \toprule
   & Futures & Cash\\
  \midrule
  May price & \$6.00/bu & \$6.00/bu \\
  \addlinespace[0.075in]
  December price & \$5.00/bu & \$5.00/bu \\
  \cmidrule(r){2-3}
  Gain/loss & \$1.00/bu & -\$1.00/bu \\
  \midrule
  Price of corn at beginning of hedge & & \textbf{\$6.00/bu} \\
  \addlinespace[0.075in]
  Gain/loss from cash position & & \textbf{-\$1.00/bu}\\
  \addlinespace[0.075in]
  Gain/loss from futures position & & \textbf{\$1.00/bu}\\
  \cmidrule(r){3-3}
  Net selling price & & \textbf{\$6.00/bu}\\
  \bottomrule
\end{tabular}
\end{table}
\begin{enumerate}[label=\textbullet]
    \item From the two examples, you can see that hedging is equivalent to locking the May price for corn of \$6.00 per bushel.
\end{enumerate}
\end{frame}


%%%%%%%%%%%%%%%%%%%%%%%%%%%%%%%%%%%%%%%%%%%%%%%%%%%%%%%%%%%%%%%%%%%%%%%%%%%%%%%%%%

\begin{frame}{Payoff lines for a short hedge}
\begin{figure}[htbp]
\begin{center}
    \begin{picture}(240,180)
        %Axises and labels
        \scriptsize
        \put(0,90){\vector(1,0){240}} %x-axis
        \put(0,0){\line(0,1){180}} %y-axis
        \put(160,80){Futures price at exercise}
        \put(-5,170){\makebox(0,0){$+$}}
        \put(-5,5){\makebox(0,0){$-$}}
        \put(-5,90){\makebox(0,0){$0$}}
        \put(-25,180){Gain}
        \put(-25,-5){Loss}
        %
        \thicklines
        \put(40,20){\vector(1,1){140}}
        \put(132,140){\vector(1,-2){9}}
        \put(115,142){Long cash}
        \color{blue}
        \put(40,160){\vector(1,-1){140}}
        \color{black}
        \put(132,40){\vector(1,2){9}}
        \put(115,33){Short futures}
        \color{red}
        \multiput(110,0)(0,5){38}{\line(0,1){1.5}}%Dashed line
        \put(100,40){\vector(1,2){9}}
        \put(60,33){Locked price}
    \end{picture}
\vspace{0.1in}
\caption{Payoff lines for a short hedge}
\end{center}
\end{figure}
\end{frame}

%%%%%%%%%%%%%%%%%%%%%%%%%%%%%%%%%%%%%%%%%%%%%%%%%%%%%%%%%%%%%%%%%%%%%%%%%%%%%%%%%%

\begin{frame}{Example: short hedge and negative basis}
\begin{enumerate}[label=\textbullet]
  \item Recall that the basis is the difference between the cash price and the futures price.
  \item Let's consider the same example as above but this time, the basis is negative:
      \begin{enumerate}[label=-]
            \item Suppose that in May, the basis is -\$0.50 per bushel.
            \item Suppose that you are aware that the basis in December is typically -\$0.15 per bushel. Let's assume that this is the basis in December.
            \item That is, the basis increases over time such that the market is in contango.
      \end{enumerate}
  \item The next slide shows gains and losses.
\end{enumerate}
\end{frame}

%%%%%%%%%%%%%%%%%%%%%%%%%%%%%%%%%%%%%%%%%%%%%%%%%%%%%%%%%%%%%%%%%%%%%%%%%%%%%%%%%%

\begin{frame}{Example: short hedge with negative basis}
\begin{table}
\caption{Price of corn increases}
\scriptsize
\begin{tabular}{l c c c}
  \toprule
   & Futures & Cash & Basis \\
  \addlinespace[0.075in]
  May price & \$6.00/bu & \$5.50/bu & -\$0.50/bu \\
  \addlinespace[0.075in]
  December price & \$7.00/bu & \$6.85/bu  & -\$0.15/bu \\
  \cmidrule(r){2-4}
  Gain/loss & -\$1.00/bu & \$1.35/bu & \$0.35/bu \\
  \midrule
  \multicolumn{2}{r}{Price of corn at beginning of hedge} & & \textbf{\$5.50/bu} \\
  \addlinespace[0.075in]
  \multicolumn{2}{r}{Gain/loss from cash position} & & \textbf{\$1.35/bu}\\
  \addlinespace[0.075in]
  \multicolumn{2}{r}{Gain/loss from futures position} & & \textbf{-\$1.00/bu}\\
  \cmidrule(r){4-4}
  \multicolumn{2}{r}{Net selling price} & & \textbf{\$5.85/bu}\\
  \bottomrule
\end{tabular}
\end{table}
\begin{enumerate}[label=\textbullet]
    \item In this case, without a hedging strategy, your net selling price would have been \$6.85 per bushel.
    \item Here, hedging costs you -\$1.00 per bushel compared to no hedging. That is because the change in price would have been favorable to you.
\end{enumerate}
\end{frame}

%%%%%%%%%%%%%%%%%%%%%%%%%%%%%%%%%%%%%%%%%%%%%%%%%%%%%%%%%%%%%%%%%%%%%%%%%%%%%%%%%%

\begin{frame}{Example: short hedge with negative basis}
\begin{table}
\caption{Price of corn declines}
\scriptsize
\begin{tabular}{l c c c}
  \toprule
   & Futures & Cash  & Basis\\
  \addlinespace[0.075in]
  May price & \$6.00/bu & \$5.50/bu & -\$0.50/bu \\
  \addlinespace[0.075in]
  December price & \$5.00/bu & \$4.85/bu  & -\$0.15/bu \\
  \cmidrule(r){2-4}
  Gain/loss & \$1.00 & -\$0.65/bu & \$0.35/bu \\
  \midrule
  \multicolumn{2}{r}{Price of corn at beginning of hedge} & & \textbf{\$5.50/bu} \\
  \addlinespace[0.075in]
  \multicolumn{2}{r}{Gain/loss from cash position} & & \textbf{-\$0.65/bu}\\
  \addlinespace[0.075in]
  \multicolumn{2}{r}{Gain/loss from futures position} & & \textbf{\$1.00/bu}\\
  \cmidrule(r){4-4}
  \multicolumn{2}{r}{Net selling price} & & \textbf{\$5.85/bu}\\
  \bottomrule
\end{tabular}
\end{table}
\begin{enumerate}[label=\textbullet]
    \item In this case, without a hedging strategy, your net selling price would have been \$4.85 per bushel.
    \item Here, you gain \$1.00 per bushel from hedging.
    \item Again, when hedging the only variation in the price is from a change in the basis.
\end{enumerate}
\end{frame}

%%%%%%%%%%%%%%%%%%%%%%%%%%%%%%%%%%%%%%%%%%%%%%%%%%%%%%%%%%%%%%%%%%%%%%%%%%%%%%%%%%

\begin{frame}[allowframebreaks]{Example: short hedge with negative basis}
\begin{enumerate}[label=\textbullet]
  \item Suppose that the two scenarios above occur with equal probability:
      \begin{enumerate}[label=\arabic*)]
            \item That is, there is a 50\% probability that the price increases to \$7.00 per bushel;
            \item And there is a 50\% probability that the price declines to \$5.00 per bushel.
      \end{enumerate}
  \item If you do not hedge:
      \begin{enumerate}[label=-]
            \item The average cash market price in December is $0.5*\$6.85+0.5*\$4.85 = \$5.85$ per bushel.
      \end{enumerate}
  \item If you hedge:
      \begin{enumerate}[label=-]
            \item The average net price you receive in December is $0.5*\$5.85+0.5*\$5.85 = \$5.85$ per bushel.
      \end{enumerate}
  \framebreak
  \item So, whether you hedge or not, your average price is the same:
      \begin{enumerate}[label=-]
            \item This is true only because there is no transaction cost and because the expected increase in the price equals the expected decline in the price.
            \item With transaction costs, the average net price from hedging would be slightly smaller.
      \end{enumerate}
\end{enumerate}
\end{frame}

%%%%%%%%%%%%%%%%%%%%%%%%%%%%%%%%%%%%%%%%%%%%%%%%%%%%%%%%%%%%%%%%%%%%%%%%%%%%%%%%%%

\begin{frame}{Why hedge then?}
\begin{enumerate}[label=\textbullet]
  \item In the example above, the average price is the same whether you hedge or not. Then, why bother with hedging, especially that in practice that you have to pay broker fees?
  \item Without hedging, in this example, there are large variations in prices (\$4.85/bu or \$6.85/bu).
  \item When hedging, you receive a constant price of \$5.85/bu.
  \item Thus, hedging removes price risk.
  \item A risk-averse farmer will be willing to pay a premium to remove price risk.
  \item That is, a risk-averse farmer is willing to accept a lower price with certainty rather than an uncertain price that is on average higher.
  \item This is paying to avoid risk.
  \item This means that some farmers will hedge, as long as the costs of hedging are not too large.
\end{enumerate}
\end{frame}


%%%%%%%%%%%%%%%%%%%%%%%%%%%%%%%%%%%%%%%%%%%%%%%%%%%%%%%%%%%%%%%%%%%%%%%%%%%%%%%%%%

\begin{frame}{Example: short hedge with positive basis}
\begin{enumerate}[label=\textbullet]
  \item In the example above the market was in contango.
  \item Fill the tables in the next two slides for the case where the market is in normal backwardation:
      \begin{enumerate}[label=-]
            \item The basis in May is \$0.50 per bushel.
            \item You expect the basis in December to be \$0.15 per bushel.
      \end{enumerate}
\end{enumerate}
\end{frame}

%%%%%%%%%%%%%%%%%%%%%%%%%%%%%%%%%%%%%%%%%%%%%%%%%%%%%%%%%%%%%%%%%%%%%%%%%%%%%%%%%%

\begin{frame}{Example: short hedge with positive basis}
\begin{table}
\caption{Price of corn increases}
\scriptsize
\begin{tabular}{l c c c}
  \toprule
   & Futures & Cash & Basis \\
  \addlinespace[0.075in]
  May price & \$6.00/bu & \$6.50/bu & \$0.50/bu \\
  \addlinespace[0.075in]
  December price & \$7.00/bu & \$7.15/bu  & \$0.15/bu \\
  \cmidrule(r){2-4}
  Gain/loss &  &  &  \\
  \midrule
  \multicolumn{2}{r}{Price of corn at beginning of hedge} & &  \\
  \addlinespace[0.075in]
  \multicolumn{2}{r}{Gain/loss from cash position} & & \\
  \addlinespace[0.075in]
  \multicolumn{2}{r}{Gain/loss from futures position} & & \\
  \cmidrule(r){4-4}
  \multicolumn{2}{r}{Net selling price} &  & \\
  \bottomrule
\end{tabular}
\end{table}
\end{frame}

%%%%%%%%%%%%%%%%%%%%%%%%%%%%%%%%%%%%%%%%%%%%%%%%%%%%%%%%%%%%%%%%%%%%%%%%%%%%%%%%%%

\begin{frame}{Example: short hedge with positive basis}
\begin{table}
\caption{Price of corn declines}
\scriptsize
\begin{tabular}{l c c c}
  \toprule
   & Futures & Cash  & Basis\\
  \addlinespace[0.075in]
  May price & \$6.00/bu & \$6.50/bu & \$0.50/bu \\
  \addlinespace[0.075in]
  December price & \$5.00/bu & \$5.15/bu  & \$0.15/bu \\
  \cmidrule(r){2-4}
  Gain/loss &  &  &  \\
  \midrule
  \multicolumn{2}{r}{Price of corn at beginning of hedge} & &  \\
  \addlinespace[0.075in]
  \multicolumn{2}{r}{Gain/loss from cash position} & & \\
  \addlinespace[0.075in]
  \multicolumn{2}{r}{Gain/loss from futures position} & & \\
  \cmidrule(r){4-4}
  \multicolumn{2}{r}{Net selling price} &  & \\
  \bottomrule
\end{tabular}
\end{table}
\end{frame}

%%%%%%%%%%%%%%%%%%%%%%%%%%%%%%%%%%%%%%%%%%%%%%%%%%%%%%%%%%%%%%%%%%%%%%%%%%%%%%%%%%
\subsection{Long hedging with futures}

\begin{frame}{Long hedging}
\begin{enumerate}[label=\textbullet]
  \item Suppose now that you manage a plant that produces ethanol from corn grain.
  \item You are expecting deliveries of corn in the Fall and wish to hedge to protect against price risk.
  \item Suppose that the current price of corn on the December futures market is \$5.50 per bushel.
  \item Just as in the previous example, your strategy is to take a position opposite to the one you have on the cash market.
  \item As you are \textbf{short in the cash market}, you must take a \textbf{long position in the futures market}.
\end{enumerate}
\end{frame}

%%%%%%%%%%%%%%%%%%%%%%%%%%%%%%%%%%%%%%%%%%%%%%%%%%%%%%%%%%%%%%%%%%%%%%%%%%%%%%%%%%
\begin{frame}{Example: long hedge with negative basis}
\begin{enumerate}[label=\textbullet]
  \item Let's consider two cases:
      \begin{enumerate}[label=\arabic*)]
            \item In December, the price of corn has increased to \$8.00 per bushel.
            \item In December, the price of corn has declined to \$5.00 per bushel.
      \end{enumerate}
  \item Suppose that the basis in May is -\$0.75 and that you expect the basis to be -\$0.25 in December.
\end{enumerate}
\end{frame}

%%%%%%%%%%%%%%%%%%%%%%%%%%%%%%%%%%%%%%%%%%%%%%%%%%%%%%%%%%%%%%%%%%%%%%%%%%%%%%%%%%

\begin{frame}{Example: long hedge with negative basis}
\begin{table}
\caption{Price of corn increases}
\scriptsize
\begin{tabular}{l c c c}
  \toprule
   & Futures & Cash & Basis \\
  \addlinespace[0.075in]
  May price & \$5.50/bu & \$4.75/bu & -\$0.75/bu \\
  \addlinespace[0.075in]
  December price & \$8.00/bu & \$7.75/bu  & -\$0.25/bu \\
  \cmidrule(r){2-4}
  Gain/loss & \$2.50/bu & -\$3.00/bu  & \$0.50/bu \\
  \midrule
  \multicolumn{2}{r}{Price of corn at beginning of hedge} & & \textbf{\$4.75/bu} \\
  \addlinespace[0.075in]
  \multicolumn{2}{r}{Gain/loss from cash position} & & \textbf{-\$3.00/bu} \\
  \addlinespace[0.075in]
  \multicolumn{2}{r}{Gain/loss from futures position} & & \textbf{\$2.50/bu} \\
  \cmidrule(r){4-4}
  \multicolumn{2}{r}{Net buying price} &  & \textbf{\$5.25/bu} \\
  \bottomrule
\end{tabular}
\end{table}
\begin{enumerate}[label=\textbullet]
    \item From your cash and futures position, you have a net loss of \$0.50/bu.
    \item As you are short in the cash market, a loss can be thought as an increase in the price.
    \item In a long hedge, a loss means an increase in the net price while a gain means a decline in the net price.
\end{enumerate}
\end{frame}

%%%%%%%%%%%%%%%%%%%%%%%%%%%%%%%%%%%%%%%%%%%%%%%%%%%%%%%%%%%%%%%%%%%%%%%%%%%%%%%%%%

\begin{frame}{Example: long hedge with negative basis}
\begin{table}
\caption{Price of corn declines}
\scriptsize
\begin{tabular}{l c c c}
  \toprule
   & Futures & Cash & Basis \\
  \addlinespace[0.075in]
  May price & \$5.50/bu & \$4.75/bu & -\$0.75/bu \\
  \addlinespace[0.075in]
  December price & \$5.00/bu & \$4.75/bu  & -\$0.25/bu \\
  \cmidrule(r){2-4}
  Gain/loss & -\$0.50/bu & \$0/bu  & \$0.50/bu \\
  \midrule
  \multicolumn{2}{r}{Price of corn at beginning of hedge} & & \textbf{\$4.75/bu} \\
  \addlinespace[0.075in]
  \multicolumn{2}{r}{Gain/loss from cash position} & & \textbf{\$0.00/bu} \\
  \addlinespace[0.075in]
  \multicolumn{2}{r}{Gain/loss from futures position} & & \textbf{-\$0.50/bu} \\
  \cmidrule(r){4-4}
  \multicolumn{2}{r}{Net buying price} &  & \textbf{\$5.25/bu} \\
  \bottomrule
\end{tabular}
\end{table}
\begin{enumerate}[label=\textbullet]
    \item Again, the hedging strategy effectively locks the price of corn and the only change in the net price is from a change in the basis.
\end{enumerate}
\end{frame}

%%%%%%%%%%%%%%%%%%%%%%%%%%%%%%%%%%%%%%%%%%%%%%%%%%%%%%%%%%%%%%%%%%%%%%%%%%%%%%%%%%

\begin{frame}{Payoff lines for a long hedge}
\begin{figure}[htbp]
\begin{center}
    \begin{picture}(240,180)
        %Axises and labels
        \scriptsize
        \put(0,90){\vector(1,0){240}} %x-axis
        \put(0,0){\line(0,1){180}} %y-axis
        \put(160,80){Futures price at exercise}
        \put(-5,170){\makebox(0,0){$+$}}
        \put(-5,5){\makebox(0,0){$-$}}
        \put(-5,90){\makebox(0,0){$0$}}
        \put(-25,180){Gain}
        \put(-25,-5){Loss}
        %
        \thicklines
        \put(40,20){\vector(1,1){140}}
        \put(132,140){\vector(1,-2){9}}
        \put(115,142){Long futures}
        \color{blue}
        \put(40,160){\vector(1,-1){140}}
        \color{black}
        \put(132,40){\vector(1,2){9}}
        \put(115,33){Short cash}
        \color{red}
        \multiput(110,0)(0,5){38}{\line(0,1){1.5}}%Dashed line
        \put(100,40){\vector(1,2){9}}
        \put(60,33){Locked price}
    \end{picture}
\vspace{0.1in}
\caption{Payoff lines for a long hedge}
\end{center}
\end{figure}
\end{frame}


%%%%%%%%%%%%%%%%%%%%%%%%%%%%%%%%%%%%%%%%%%%%%%%%%%%%%%%%%%%%%%%%%%%%%%%%%%%%%%%%%%

\begin{frame}{Example: long hedge with negative basis}
\begin{enumerate}[label=\textbullet]
  \item Suppose that the two scenarios above occur with equal probability:
      \begin{enumerate}[label=-]
            \item That is, there is a 50\% probability that the futures price increases to \$8.00 per bushel;
            \item And there is a 50\% probability that the futures price declines to \$5.00 per bushel.
      \end{enumerate}
  \item If you do not hedge:
      \begin{enumerate}[label=-]
            \item The average cash market price in December is $0.5*\$7.75+0.5*\$4.75 = \$6.25$ per bushel.
      \end{enumerate}
  \item If you hedge:
      \begin{enumerate}[label=-]
            \item The average net price you pay in December is $0.5*\$5.25+0.5*\$5.25 = \$5.25$ per bushel.
      \end{enumerate}
  \item So, in this case, your average price is not the same when you hedge:
      \begin{enumerate}[label=-]
            \item This is because the increase in the price is larger than the decline in the price.
      \end{enumerate}
\end{enumerate}
\end{frame}

%%%%%%%%%%%%%%%%%%%%%%%%%%%%%%%%%%%%%%%%%%%%%%%%%%%%%%%%%%%%%%%%%%%%%%%%%%%%%%%%%%

\begin{frame}{Why hedging?}
\begin{enumerate}[label=\textbullet]
  \item In the example above, as a buyer of the commodity, you obviously gain on average from hedging (pay lower price).
  \item This is because the risk of an increase in price is larger than the risk of a decline in price.
\end{enumerate}
\end{frame}

%%%%%%%%%%%%%%%%%%%%%%%%%%%%%%%%%%%%%%%%%%%%%%%%%%%%%%%%%%%%%%%%%%%%%%%%%%%%%%%%%%

\begin{frame}{Why hedging?}
\begin{enumerate}[label=\textbullet]
  \item Suppose instead that there is a 90\% chance that the futures price declines to \$5.00 per bushel and a 10\% chance that it increases to \$8.00 per bushel.
      \begin{enumerate}[label=-]
            \item If you do not hedge, the average price that you pay in December is $0.1*\$7.75+0.9*\$4.75 = \$5.05$ per bushel.
            \item If you hedge, the average net price you pay in December is $0.1*\$5.25+0.9*\$5.25 = \$5.25$ per bushel.
            \item In this case, the hedging strategy yields on average a higher price.
            \item Still, if you are risk averse you may find it beneficial to hedge your price risk.
      \end{enumerate}
\end{enumerate}
\end{frame}

%%%%%%%%%%%%%%%%%%%%%%%%%%%%%%%%%%%%%%%%%%%%%%%%%%%%%%%%%%%%%%%%%%%%%%%%%%%%%%%%%%
\begin{frame}{Example: long hedge with positive basis}
\begin{enumerate}[label=\textbullet]
  \item Suppose that the futures price for corn is \$5.75/bu
  \item Consider two cases:
      \begin{enumerate}[label=\arabic*)]
            \item In December, the futures price for corn has increased to \$7.50 per bushel.
            \item In December, the futures price for corn has declined to \$4.50 per bushel.
      \end{enumerate}
  \item Suppose that the basis in May is \$0.50 and that you expect the basis to be \$0.05 in December.
  \item Fill the following two tables.
\end{enumerate}
\end{frame}

%%%%%%%%%%%%%%%%%%%%%%%%%%%%%%%%%%%%%%%%%%%%%%%%%%%%%%%%%%%%%%%%%%%%%%%%%%%%%%%%%%

\begin{frame}{Example: long hedge with positive basis}
\begin{table}
\caption{Price of corn increases}
\scriptsize
\begin{tabular}{l c c c}
  \toprule
   & Futures & Cash & Basis \\
  \addlinespace[0.075in]
  May price &  &  &  \\
  \addlinespace[0.075in]
  December price &  &   &  \\
  \cmidrule(r){2-4}
  Gain/loss &  &   &  \\
  \midrule
  \multicolumn{2}{r}{Price of corn at beginning of hedge} & &  \\
  \addlinespace[0.075in]
  \multicolumn{2}{r}{Gain/loss from cash position} & &  \\
  \addlinespace[0.075in]
  \multicolumn{2}{r}{Gain/loss from futures position} & &  \\
  \cmidrule(r){4-4}
  \multicolumn{2}{r}{Net buying price} &  & \\
  \bottomrule
\end{tabular}
\end{table}
\end{frame}

%%%%%%%%%%%%%%%%%%%%%%%%%%%%%%%%%%%%%%%%%%%%%%%%%%%%%%%%%%%%%%%%%%%%%%%%%%%%%%%%%%

\begin{frame}{Example: long hedge with positive basis}
\begin{table}
\caption{Price of corn declines}
\scriptsize
\begin{tabular}{l c c c}
  \toprule
   & Futures & Cash & Basis \\
  \addlinespace[0.075in]
  May price &  &  &  \\
  \addlinespace[0.075in]
  December price &  &   &  \\
  \cmidrule(r){2-4}
  Gain/loss &  &   &  \\
  \midrule
  \multicolumn{2}{r}{Price of corn at beginning of hedge} & &  \\
  \addlinespace[0.075in]
  \multicolumn{2}{r}{Gain/loss from cash position} & &  \\
  \addlinespace[0.075in]
  \multicolumn{2}{r}{Gain/loss from futures position} & &  \\
  \cmidrule(r){4-4}
  \multicolumn{2}{r}{Net buying price} &  & \\
  \bottomrule
\end{tabular}
\end{table}
\end{frame}

%%%%%%%%%%%%%%%%%%%%%%%%%%%%%%%%%%%%%%%%%%%%%%%%%%%%%%%%%%%%%%%%%%%%%%%%%%%%%%%%%%
\begin{frame}{Example: long hedge with positive basis}
\begin{enumerate}[label=\textbullet]
  \item Calculate the average price if you do not hedge.
  \bigskip \bigskip
  \item Calculate the average price if you decide to hedge.
  \bigskip \bigskip
  \item Which strategy is best based solely on the average price (ignore preferences regarding risk)?
\end{enumerate}
\end{frame}

%%%%%%%%%%%%%%%%%%%%%%%%%%%%%%%%%%%%%%%%%%%%%%%%%%%%%%%%%%%%%%%%%%%%%%%%%%%%%%%%%%
\subsection{Hedging and the basis}

\begin{frame}{Information about the basis in Iowa}
\begin{enumerate}[label=\textbullet]
  \item So far, we have looked at examples where we assumed knowledge of how the basis evolves during the hedge.
  \item In practice, there is no sure way of knowing how the basis changes over time.
  \item There are however seasonal patterns in the basis that it is possible to use in an effort to forecast the basis over time.
  \item This means that there is still some risk from hedging.
  \item In fact, once the position is taken on the futures market, all the risk that remains is from the basis.
  \item From the examples above, notice that the net price from a hedging position equals the sum of the cash price at the moment of taking the position and the change in the basis.
      \begin{enumerate}[label=-]
            \item That is true both for the short and long position.
      \end{enumerate}
\end{enumerate}
\end{frame}

%%%%%%%%%%%%%%%%%%%%%%%%%%%%%%%%%%%%%%%%%%%%%%%%%%%%%%%%%%%%%%%%%%%%%%%%%%%%%%%%%%

\begin{frame}{Information about the basis in Iowa}
\begin{enumerate}[label=\textbullet]
    \item Recall that we can find information about the average basis over time in Iowa from the Extension service at Iowa State.
    \item The webpage is available \href{http://www.extension.iastate.edu/agdm/crops/html/a2-41.html}{here}.
\end{enumerate}
\end{frame}

%%%%%%%%%%%%%%%%%%%%%%%%%%%%%%%%%%%%%%%%%%%%%%%%%%%%%%%%%%%%%%%%%%%%%%%%%%%%%%%%%%

\begin{frame}{Effect of basis on hedging}
\begin{enumerate}[label=\textbullet]
    \item The examples above show you the effect of the basis on hedging.
    \item Let's look at the effect of the basis in more detail.
    \item Suppose that the price of corn is currently \$6.00 per bushel and that the basis is -\$0.25 per bushel such that the current cash price is \$5.75.
\end{enumerate}
\end{frame}

%%%%%%%%%%%%%%%%%%%%%%%%%%%%%%%%%%%%%%%%%%%%%%%%%%%%%%%%%%%%%%%%%%%%%%%%%%%%%%%%%%

\begin{frame}{Net price from changes in basis}
    \framedgraphic{hedging_basis.png}
\end{frame}

%%%%%%%%%%%%%%%%%%%%%%%%%%%%%%%%%%%%%%%%%%%%%%%%%%%%%%%%%%%%%%%%%%%%%%%%%%%%%%%%%%

\begin{frame}{Effect of basis on hedging}
\begin{enumerate}[label=\textbullet]
    \item You can see from the previous graph:
        \begin{enumerate}[label=-]
            \item If \textbf{the basis strengthens} (e.g. goes from -0.35 to -0.25), then \textbf{the net price increases}.
            \item If \textbf{the basis weakens} (e.g. goes from 0.25 to -0.05), then \textbf{the net price declines}.
        \end{enumerate}
    \item Intuitively, if the local cash market price increases by more than the futures price, then the net price from the hedging position increases.
\end{enumerate}
\end{frame}

%%%%%%%%%%%%%%%%%%%%%%%%%%%%%%%%%%%%%%%%%%%%%%%%%%%%%%%%%%%%%%%%%%%%%%%%%%%%%%%%%%

\begin{frame}{Basis risk and hedging}
\begin{enumerate}[label=\textbullet]
    \item Hedging still presents some risk because of uncertainty in the basis.
    \item However, basis risk should be much smaller than price risk.
    \item After all, the basis should be a relatively small share of the price and should be bounded by transaction costs and arbitrage opportunities.
\end{enumerate}
\end{frame}

%%%%%%%%%%%%%%%%%%%%%%%%%%%%%%%%%%%%%%%%%%%%%%%%%%%%%%%%%%%%%%%%%%%%%%%%%%%%%%%%%%

\begin{frame}{Summary: the basis and hedging with futures}
\begin{table}
\caption{Impact of change in basis on hedger's revenue}
\begin{tabular}{l c c}
  \toprule
  & \multicolumn{2}{c}{Change in the basis over hedge period}\\
  \cmidrule(r){2-3}
  Type of hedge & Stronger basis & Weaker basis\\
  \midrule
   Short hedge & Favorable  & Unfavorable \\
   Long hedge &  Unfavorable & Favorable \\
  \bottomrule
\end{tabular}
\end{table}
\end{frame}

%%%%%%%%%%%%%%%%%%%%%%%%%%%%%%%%%%%%%%%%%%%%%%%%%%%%%%%%%%%%%%%%%%%%%%%%%%%%%%%%%%
\subsection{Direction of expected price change and hedging decision}

\begin{frame}{Risk direction and decision to hedge}
\begin{enumerate}[label=\textbullet]
  \item We saw that hedging effectively locks a price, which is not too far from the current futures price depending on how the basis evolves over time.
  \item Based on the current price of a futures contract, when is it best to enter into hedging?
      \begin{enumerate}[label=-]
            \item If you are \textbf{long in the cash market}, it is best for you to hedge when the\textbf{ current futures price is high} and your expectations are that the \textbf{price of a commodity will decline}.
            \item If you are \textbf{short in the cash market}, it is best for you to hedge when the\textbf{ current futures price is low} and your expectations are that the \textbf{price of a commodity will increase}.
      \end{enumerate}
\end{enumerate}
\end{frame}


%%%%%%%%%%%%%%%%%%%%%%%%%%%%%%%%%%%%%%%%%%%%%%%%%%%%%%%%%%%%%%%%%%%%%%%%%%%%%%%%%%
\subsection{Choice of market and futures contract for hedging}

\begin{frame}[allowframebreaks]{Which market and futures contract?}
\begin{enumerate}[label=\textbullet]
  \item It is easier to trade on exchanges with large volumes, in particular, it is easier to liquidate a position in exchanges that trade large volumes. Prices in thin markets are more volatile.
  \item The futures contract must cover the entire duration of the hedge. A good idea is to choose a futures contract that expires shortly after the hedge such that a single futures contract can be used for the duration of the hedge.
  \item It is possible to hedge by rolling over from one futures contract to another futures contract. That however requires paying more in transaction fees.
  \item Ideally, you want to chose a futures contract for which the basis evolves over time in a way that is predictable and favorable to you.
  \item In a simple hedge, the definition of the commodity in the futures contract should be as closed as possible to the commodity of interest. This assures a strong correlation between the market price and the futures price.
  \item It is possible to hedge with a different commodity. This is called a \emph{cross-hedge}.
            \begin{enumerate}[label=-]
            \item For a cross-hedge, the price on the futures market of the other commodity used for the hedge should have a strong positive correlation with the commodity for which you wish to reduce price risk.
            \item A strong negative correlation is also ok but the hedging position must be the inverse of the hedging strategy that we have covered in this section.
      \end{enumerate}
\end{enumerate}
\end{frame}

%%%%%%%%%%%%%%%%%%%%%%%%%%%%%%%%%%%%%%%%%%%%%%%%%%%%%%%%%%%%%%%%%%%%%%%%%%%%%%%%%%%%
\section[References]{References}
\renewcommand\refname{References}
\def\newblock{References}
\begin{frame}[allowframebreaks]{References}
\bibliography{D:/Dropbox/Papers/References}
%\bibliography{D:/Papers/References}
\end{frame}


%%%%%%%%%%%%%%%%%%%%%%%%%%%%%%%%%%%%%%%%%%%%%%%%%%%%%%%%%%%%%%%%%%%%%%%%%%%%%%%%%%%%%

\end{document}
