%\documentclass[table,xcolor=pdftex,dvipsnames]{beamer}

\documentclass[table,xcolor=pdftex,dvipsnames, handout]{beamer}\usepackage[]{graphicx}\usepackage[]{color}
%% maxwidth is the original width if it is less than linewidth
%% otherwise use linewidth (to make sure the graphics do not exceed the margin)
\makeatletter
\def\maxwidth{ %
  \ifdim\Gin@nat@width>\linewidth
    \linewidth
  \else
    \Gin@nat@width
  \fi
}
\makeatother

\definecolor{fgcolor}{rgb}{0.345, 0.345, 0.345}
\newcommand{\hlnum}[1]{\textcolor[rgb]{0.686,0.059,0.569}{#1}}%
\newcommand{\hlstr}[1]{\textcolor[rgb]{0.192,0.494,0.8}{#1}}%
\newcommand{\hlcom}[1]{\textcolor[rgb]{0.678,0.584,0.686}{\textit{#1}}}%
\newcommand{\hlopt}[1]{\textcolor[rgb]{0,0,0}{#1}}%
\newcommand{\hlstd}[1]{\textcolor[rgb]{0.345,0.345,0.345}{#1}}%
\newcommand{\hlkwa}[1]{\textcolor[rgb]{0.161,0.373,0.58}{\textbf{#1}}}%
\newcommand{\hlkwb}[1]{\textcolor[rgb]{0.69,0.353,0.396}{#1}}%
\newcommand{\hlkwc}[1]{\textcolor[rgb]{0.333,0.667,0.333}{#1}}%
\newcommand{\hlkwd}[1]{\textcolor[rgb]{0.737,0.353,0.396}{\textbf{#1}}}%
\let\hlipl\hlkwb

\usepackage{framed}
\makeatletter
\newenvironment{kframe}{%
 \def\at@end@of@kframe{}%
 \ifinner\ifhmode%
  \def\at@end@of@kframe{\end{minipage}}%
  \begin{minipage}{\columnwidth}%
 \fi\fi%
 \def\FrameCommand##1{\hskip\@totalleftmargin \hskip-\fboxsep
 \colorbox{shadecolor}{##1}\hskip-\fboxsep
     % There is no \\@totalrightmargin, so:
     \hskip-\linewidth \hskip-\@totalleftmargin \hskip\columnwidth}%
 \MakeFramed {\advance\hsize-\width
   \@totalleftmargin\z@ \linewidth\hsize
   \@setminipage}}%
 {\par\unskip\endMakeFramed%
 \at@end@of@kframe}
\makeatother

\definecolor{shadecolor}{rgb}{.97, .97, .97}
\definecolor{messagecolor}{rgb}{0, 0, 0}
\definecolor{warningcolor}{rgb}{1, 0, 1}
\definecolor{errorcolor}{rgb}{1, 0, 0}
\newenvironment{knitrout}{}{} % an empty environment to be redefined in TeX

\usepackage{alltt}
\usepackage{handoutWithNotes}
\pgfpagesuselayout{4 on 1 with notes}[letterpaper,border shrink=5mm]

\usepackage{beamerthemesplit}
\usepackage[english]{babel}
\usepackage{amsmath}
\usepackage{amssymb}
\usepackage{amsthm}
\usepackage{verbatim}
\usepackage{graphpap}
\usepackage{epic}
\usepackage{pict2e} %To draw line with any slope
\usepackage{color}
\usepackage{natbib}
\usepackage{enumitem}
\usepackage{booktabs}
\usepackage{xcolor}
\usepackage{textcomp}
%\usepackage{movie15}

\bibliographystyle{ajae}

\newcommand{\p}{\partial}

\newcommand {\framedgraphic}[1] {
        \begin{center}
            \includegraphics[width=\textwidth,height=0.7\textheight,keepaspectratio]{#1}
        \end{center}
        \vspace{-1\baselineskip}
}

\usetheme{Boadilla}
\useoutertheme{shadow}
\usecolortheme{beaver}%seagull
\everymath{\color{blue}}
\everydisplay{\color{blue}}

\usefonttheme{professionalfonts}

\AtBeginDocument{%
\setlength{\abovecaptionskip}{3pt}
\setlength{\belowcaptionskip}{3pt}
\setlength{\floatsep}{0pt}
\setlength{\textfloatsep}{3pt}
\setlength{\intextsep}{3pt}
}
\usepackage{hyperref}
\hypersetup{
   colorlinks = {true},
   urlcolor = {blue},
   linkcolor = {black},
   citecolor = {black},
   pdfborderstyle={/S/U/W 1},
   urlbordercolor = 0 0 1,
   citebordercolor = 1 1 1,
   filebordercolor = 1 1 1,
   linkbordercolor = 1 1 1,
   pdfauthor = {Sebastien Pouliot},
}

\widowpenalty=10000 % Avoid single line at the end of a page
\clubpenalty=10000  % Avoid single line at the bottom

\title[Commodity storage]{Commodity storage}
\author[Pouliot]{S\'{e}bastien Pouliot}
\institute{Iowa State University}
\date{Fall 2017}
\IfFileExists{upquote.sty}{\usepackage{upquote}}{}
\begin{document}

%%%%%%%%%%%%%%%%%%%%%%%%%%%%%%%%%%%%%%%%%%%%%%%%%%%%%%%%%%%%%%%%%%%%%%%%%%%%%%%%%%

\begin{frame}
\titlepage
\vspace{-0.4in}
\begin{center}
Lecture notes for Econ 235\\
\end{center}
\end{frame}

%%%%%%%%%%%%%%%%%%%%%%%%%%%%%%%%%%%%%%%%%%%%%%%%%%%%%%%%%%%%%%%%%%%%%%%%%%%%%%%%%%
\section{Introduction}

\begin{frame}{Commodity storage}
\begin{enumerate}[label=\textbullet]
  \item Storage plays an important in stabalizing prices through time.
  \item The law of one price also applies through time.
      \begin{enumerate}[label=-]
          \item  Remember that the \emph{law of one price} says that there is one price for a commodity once accounting for transaction costs.
       \end{enumerate}  
  \item Storage allows for arbitrage between the market today and the market at a later time
  \item Examples of transaction costs through time:
    \begin{enumerate}[label=-]
      \item Interest rate;
      \item Storage;
      \item Spoilage.
   \end{enumerate}
\end{enumerate}
\end{frame}

%%%%%%%%%%%%%%%%%%%%%%%%%%%%%%%%%%%%%%%%%%%%%%%%%%%%%%%%%%%%%%%%%%%%%%%%%%%%%%%%%

\begin{frame}{Resources to understand the storage}
\begin{enumerate}[label=\textbullet]
  \item \href{https://www.extension.iastate.edu/agdm/crops/pdf/a2-33.pdf}{Cost of storing grain} from ISU extension.
  \item \href{http://mindymallory.github.io/PriceAnalysis/prices-over-space-and-time.html}{Prices over space and time} from Mindy Mallory textbook.
\end{enumerate}
\end{frame}


%%%%%%%%%%%%%%%%%%%%%%%%%%%%%%%%%%%%%%%%%%%%%%%%%%%%%%%%%%%%%%%%%%%%%%%%%%%%%%%%%%

\begin{frame}{Storage decision}
\begin{enumerate}[label=\textbullet]
  \item To understand the decision to store a commodity, we must compare the price of a commodity today and the expected price for that commodity at a later date.
  \item Write the price of a commodity today as $P_t$.
  \item Write today's expectations of the price of the commodity in the next period (e.g. next month) is $E_t P_{t+1}$.
\end{enumerate}
\end{frame}

%%%%%%%%%%%%%%%%%%%%%%%%%%%%%%%%%%%%%%%%%%%%%%%%%%%%%%%%%%%%%%%%%%%%%%%%%%%%%%%%%%

\begin{frame}{Price expectations}
\begin{enumerate}[label=\textbullet]
    \item Nobody knows the exact price in the next period but everybody can have expectations regarding the price in the next period.
    \item Market participants can form their expectations about prices in the future using different methods:
    \begin{enumerate}[label=-]
      \item Price paid in the current period;
      \item Projections from economists or other market analysts;
      \item Futures prices;
      \item Other heuristics.
    \end{enumerate}
    \item The futures price represents what the market thinks the price in the next period will be such that we can write $E_t P_{t+1} = F_t$, where $F_t$ is the current price of a futures contract that expires in the next period.
    \item However, an individual trader might not agree and his/her expectations of the future might not be the same as the market. For such a trader, is it possible that $E_t P_{t+1} >F_t$ or $E_t P_{t+1} < F_t$.
\end{enumerate}
\end{frame}

%%%%%%%%%%%%%%%%%%%%%%%%%%%%%%%%%%%%%%%%%%%%%%%%%%%%%%%%%%%%%%%%%%%%%%%%%%%%%%%%%%

\begin{frame}{Cost of storage}
\begin{enumerate}[label=\textbullet]
    \item Write that the unit-cost of storage per period for a commodity is $s$.
        \begin{enumerate}[label=-]
          \item This includes all costs such as the physical cost of storage, spoilage and insurance.
          \item Assume that it is constant through time (i.e. $s_t = s$).
          \item \href{https://www.extension.iastate.edu/agdm/crops/pdf/a2-33.pdf}{Cost of storing grain} provides values for the cost of storing grains. You can also use an excel sheet to check on the \href{https://www.extension.iastate.edu/agdm/cdmarkets.html}{profitability} of storing grain.
       \end{enumerate}
    \item Write the interest rate for one period as $r$.
        \begin{enumerate}[label=-]
          \item The interest rate is the opportunity cost of holding money over time.
          \item One dollar invested today is worth $1+r$ dollars in the next period.
          \item Or, one dollar in the next period is worth $\beta = \frac{1}{1+r} < 1$ dollars today.
       \end{enumerate}
\end{enumerate}
\end{frame}

%%%%%%%%%%%%%%%%%%%%%%%%%%%%%%%%%%%%%%%%%%%%%%%%%%%%%%%%%%%%%%%%%%%%%%%%%%%%%%%%%%

\begin{frame}{Convenience yield}
\begin{enumerate}[label=\textbullet]
   \item Write that $c_t$ is the \emph{convenience yield} from holding the commodity for one more period.
       \begin{enumerate}[label=-]
         \item It represents the gains from the flow of services from holding one unit of the commodity to the next period.
         \item This is a very broad term that captures all sorts of gains from holding the commodity, including the gains from reducing uncertainty and breeding in the case of livestock.
         \item As an example of the existence of a convenience yield, we can look at the \href{http://usda.mannlib.cornell.edu/usda/waob/wasde//2010s/2013/wasde-08-12-2013.pdf}{2012-13 crop year}.
      \end{enumerate}
   \item There is good evidence that the convenience yield exists. To simplify here, assume that equals zero (i.e. $c_t=0$).
\end{enumerate}
\end{frame}


%%%%%%%%%%%%%%%%%%%%%%%%%%%%%%%%%%%%%%%%%%%%%%%%%%%%%%%%%%%%%%%%%%%%%%%%%%%%%%%%%%

\begin{frame}{Return to storage}
\begin{enumerate}[label=\textbullet]
    \item The return from selling one unit of the commodity today is $P_t$.
    \item The expected return from selling one unit of the commodity in the next period is $\beta (E_t P_{t+1} - s)$.
    \item That is, the expected return is the expected price, minus the cost of storage for one period, all in today's dollars.
\end{enumerate}
\end{frame}


%%%%%%%%%%%%%%%%%%%%%%%%%%%%%%%%%%%%%%%%%%%%%%%%%%%%%%%%%%%%%%%%%%%%%%%%%%%%%%%%%%

\begin{frame}{Temporal arbitrage}
\begin{enumerate}[label=\textbullet]
    \item For the holder of a storable commodity, the question is ``sell today or sell later?''
    \item Should sell today if the expected gain is larger then selling in the next period: \[ P_t \ge \beta (E_t P_{t+1} - s). \]
    \vspace{-1\baselineskip}
    \item Should sell later if the expected gain is larger then selling today: \[ \beta (E_t P_{t+1} - s) \ge P_t. \]
    \vspace{-1\baselineskip}
    \item The two inequalities above are the intertemporal arbitrage conditions.
\end{enumerate}
\end{frame}


%%%%%%%%%%%%%%%%%%%%%%%%%%%%%%%%%%%%%%%%%%%%%%%%%%%%%%%%%%%%%%%%%%%%%%%%%%%%%%%%%%

\begin{frame}{Temporal arbitrage - example}
\begin{enumerate}[label=\textbullet]
    \item Suppose that you can sell corn today at a price of \$3.25/bu.
    \item You consider storing a certain quantity of corn. The futures price for corn in March is \$3.90/bu and you expect that the basis in March will be -\$0.25/bu.
    \item Your cost of storage from the harvest (October) until March is \$0.35/bu.
    \item The annual interest rate is 6\%.
    \item Given those expectations, should you store corn?
\end{enumerate}
\end{frame}

%%%%%%%%%%%%%%%%%%%%%%%%%%%%%%%%%%%%%%%%%%%%%%%%%%%%%%%%%%%%%%%%%%%%%%%%%%%%%%%%%%

\begin{frame}{Temporal arbitrage - example}
\begin{enumerate}[label=\textbullet]
    \item You expect that you'll be able to sell corn at \$3.65/bu = \$3.90/bu -\$0.25/bu.
    \item The interest rate from October to March (5 months) is $0.06*5/12 = 0.025$.
    \item A dollar in March is worth $0.9756 = 1/(1+0.025)$.
    \item Your expected return is $0.9756 *(\$3.65/bu - \$0.35/bu)= \$3.22/bu$.
    \item Given those expectations, should you store corn?
\end{enumerate}
\end{frame}

%%%%%%%%%%%%%%%%%%%%%%%%%%%%%%%%%%%%%%%%%%%%%%%%%%%%%%%%%%%%%%%%%%%%%%%%%%%%%%%%%%

\begin{frame}{Temporal arbitrage - example}
\begin{enumerate}[label=\textbullet]
    \item The answer is no, because you can earn \$0.03/bu (\$3.25/bu - \$3.22/bu) more by selling now versus selling in March.
    \item You can verify that if you storage cost is \$0.25/bu that it would be profitable for you to store corn between October and March.
\end{enumerate}
\end{frame}

%%%%%%%%%%%%%%%%%%%%%%%%%%%%%%%%%%%%%%%%%%%%%%%%%%%%%%%%%%%%%%%%%%%%%%%%%%%%%%%%%%

\begin{frame}{No-arbitrage temporal condition}
\begin{enumerate}[label=\textbullet]
    \item If intertemporal arbitrage holds, it means that \[  P_t = \beta (E_t P_{t+1} - s). \]
    \vspace{-1\baselineskip}
    \item What this expression says is that, because of arbitrage, the market pays the holder of a commodity his cost of carrying the commodity for one period.
\end{enumerate}
\end{frame}


%%%%%%%%%%%%%%%%%%%%%%%%%%%%%%%%%%%%%%%%%%%%%%%%%%%%%%%%%%%%%%%%%%%%%%%%%%%%%%%%%%

\begin{frame}{The basis through time}
\begin{enumerate}[label=\textbullet]
    \item Consider the cash price (right now) and the futures price (or the price under a forward contract) for a commodity.
    \item Remember that the price of a futures contract represents what the market thinks the price of a commodity will be at the expiration of the contract (i.e. $E_t P_{t+1} = F_t$).
    \item To focus on how the basis changes through time, \textbf{assume for now that the futures contract specifies a delivery location that is the same as the local cash market.} Thus, we assume for now that there are no spatial issues.
    \item Over time, the price of a futures and the cash price will tend to vary together because of intertemporal arbitrage -  if possible.
    \item However, will the basis be positive or negative?
    \item The only thing that we know for sure is that the basis will be zero when the futures contract expires.
\end{enumerate}
\end{frame}

%%%%%%%%%%%%%%%%%%%%%%%%%%%%%%%%%%%%%%%%%%%%%%%%%%%%%%%%%%%%%%%%%%%%%%%%%%%%%%%%%%

\begin{frame}{The basis through time for a storable commodity}
\begin{enumerate}[label=\textbullet]
    \item For a storable commodity, intertemporal arbitrage is possible.
    \item This means that the holder of a commodity will make the decision to sell today or wait to sell later.
    \item This is possible because of storage.
    \item We can find a relationship in the price over time from the cost of storage and the cost of money (interest rate).
\end{enumerate}
\end{frame}

%%%%%%%%%%%%%%%%%%%%%%%%%%%%%%%%%%%%%%%%%%%%%%%%%%%%%%%%%%%%%%%%%%%%%%%%%%%%%%%%%%

\begin{frame}{Prices through time for a storable commodity}
\begin{enumerate}[label=\textbullet]
    \item The ability to arbitrage through time for storable commodities creates a relationship between prices over time.
        \begin{enumerate}[label=-]
          \item If \[  P_t = \beta (F_t - s),\]
            \vspace{-1\baselineskip}
          \item and that \[ E_t P_{t+1} = \beta (F_{t+1} - s), \]
            \vspace{-1\baselineskip}
          \item it means that \[ P_t < F_t  < F_{t+1} < ... \]
            \vspace{-1\baselineskip}
        \end{enumerate}
    \item That is, as time goes by, the price of a storable commodity should increase to compensate for the opportunity cost of holding the commodity (cost of storage).
\end{enumerate}
\end{frame}

%%%%%%%%%%%%%%%%%%%%%%%%%%%%%%%%%%%%%%%%%%%%%%%%%%%%%%%%%%%%%%%%%%%%%%%%%%%%%%%%%%

\begin{frame}{The basis through time for a storable commodity}
\begin{enumerate}[label=\textbullet]
    \item The further from the expiration of a futures contract, the more negative is the basis.
    \item The implication is that the basis will narrow over time when approaching the expiration date of a contract.
    \item The relationship between prices at different periods may not hold all the time:
        \begin{enumerate}[label=-]
          \item In particular, the relationship may not appear to hold because of changes in expectations regarding future prices. For example, a sudden shock may affect $F_t$.
          \item It may not hold because of different crop seasons.
        \end{enumerate}
\end{enumerate}
\end{frame}

%%%%%%%%%%%%%%%%%%%%%%%%%%%%%%%%%%%%%%%%%%%%%%%%%%%%%%%%%%%%%%%%%%%%%%%%%%%%%%%%%%

\begin{frame}{Example of basis for corn through time}\label{Corn_basis}
    \framedgraphic{Corn_basis.png}
Source: \href{http://www.keycoop.com/}{Key cooperative}.
\end{frame}

%%%%%%%%%%%%%%%%%%%%%%%%%%%%%%%%%%%%%%%%%%%%%%%%%%%%%%%%%%%%%%%%%%%%%%%%%%%%%%%%%%

\begin{frame}{Information about the basis in Iowa}
\begin{enumerate}[label=\textbullet]
    \item The Extension service here at Iowa State compiled data regarding the basis.
    \item The webpage is available \href{http://www.extension.iastate.edu/agdm/crops/html/a2-41.html}{here}.
    \item \href{http://www.ams.usda.gov/mnreports/nw_gr110.txt}{AMS} publishes regional bases for Iowa.
\end{enumerate}
\end{frame}

%%%%%%%%%%%%%%%%%%%%%%%%%%%%%%%%%%%%%%%%%%%%%%%%%%%%%%%%%%%%%%%%%%%%%%%%%%%%%%%%%%

\subsubsection{Definitions: contango and normal backwardation}

\begin{frame}{Definition: contango}
\begin{enumerate}[label=\textbullet]
    \item A market is in \emph{contango} if the price of futures contract is higher than the price on the spot market.
    \item This is also referred as a \emph{positive carrying charge market}.
    \item This is a direct consequence of intertemporal arbitrage described earlier.
    \item See futures prices on \href{http://www.barchart.com/futures/marketoverview}{Barchart} (not all are good examples).
    \item In the same crop year, the difference between futures price and the spot price reflects the price of storage.
\end{enumerate}
\end{frame}


%%%%%%%%%%%%%%%%%%%%%%%%%%%%%%%%%%%%%%%%%%%%%%%%%%%%%%%%%%%%%%%%%%%%%%%%%%%%%%%%%%

\begin{frame}{Definition: backwardation}
\begin{enumerate}[label=\textbullet]
    \item The price pattern between cash and futures prices described before may sometimes not hold.
    \item A situation where the cash price is higher than the price of a futures contract is referred to as \emph{negative carrying charge} or \emph{normal backwardation}.
\end{enumerate}
\end{frame}

%%%%%%%%%%%%%%%%%%%%%%%%%%%%%%%%%%%%%%%%%%%%%%%%%%%%%%%%%%%%%%%%%%%%%%%%%%%%%%%%%%

\begin{frame}{Backwardation}
\begin{enumerate}[label=\textbullet]
    \item Backwardation occurs when  \[ P_t \ge \beta (E_t P_{t+1} - s). \]
            \vspace{-1\baselineskip}
    \item Backwardation is considered abnormal: how can it be sustained if commodity traders arbitrage between periods?
    \item One explanation for backwardation is that there is a small number of transactions. In such case, a risk premium might explain the inversion.%page 91 in Carter
        \begin{enumerate}[label=-]
          \item In a thin market, buyers may be willing to pay a premium to secure supply early because they fear that they will only be able to buy the commodity in the months ahead at a much higher price later.
          \item The market is not paying for the cost of storage.
        \end{enumerate}
    \item Another explanation is a short-term shortage of deliverable stocks (i.e. there is no longer any stock available).
\end{enumerate}
\end{frame}


%%%%%%%%%%%%%%%%%%%%%%%%%%%%%%%%%%%%%%%%%%%%%%%%%%%%%%%%%%%%%%%%%%%%%%%%%%%%%%%%%%

\begin{frame}{Examples of arbitrage through time: storable commodity}
\begin{enumerate}[label=\textbullet]
    \item Hay bales;
    \item Grain elevators;
    \item Livestock (storing meat on hooves - possible but not for very long);
    \item Apples (perishable but storable for about 8 months);
    \item Canning;
    \item Beer and cheese are examples of storage.
\end{enumerate}
\end{frame}

%%%%%%%%%%%%%%%%%%%%%%%%%%%%%%%%%%%%%%%%%%%%%%%%%%%%%%%%%%%%%%%%%%%%%%%%%%%%%%%%%%

\begin{frame}{What about non-storable commodities?}
\begin{enumerate}[label=\textbullet]
    \item For a non-storable commodity, there is no intertemporal arbitrage:
        \begin{enumerate}[label=-]
          \item One cannot hold onto a non-storable commodity in hope of selling it a higher price later.
          \item For examples, fresh fruits can be stored but only for a short period of time.
       \end{enumerate}
    \item Thus, for non-storable commodities, the basis depends on the current market conditions and the expected conditions and the time of delivery for the futures contract.
    \item The markets for the commodity now and in the future are different.
    \item There is no link between the market today and the market in the future.
\end{enumerate}
\end{frame}


%%%%%%%%%%%%%%%%%%%%%%%%%%%%%%%%%%%%%%%%%%%%%%%%%%%%%%%%%%%%%%%%%%%%%%%%%%%%%%%%%%
\section{Summary}

\begin{frame}{In summary...}
\begin{enumerate}[label=\textbullet]
    \item The basis for a futures contract has both a time and spatial component.
    \item Thinking about the basis through time or through space is the same as it is the transaction cost that determine the basis:
        \begin{enumerate}[label=-]
          \item Opportunity cost of holding a commodity through time;
          \item Cost of transportation from one location to another;
          \item Time only moves in one direction.
        \end{enumerate}
    \item The cash and futures prices may not converge exactly to the same value because:
        \begin{enumerate}[label=-]
          \item The local cash market is not a terminal market;
          \item Difference in quality;
          \item Shocks for which the market takes time to adjust to.
        \end{enumerate}
\end{enumerate}
\end{frame}


%%%%%%%%%%%%%%%%%%%%%%%%%%%%%%%%%%%%%%%%%%%%%%%%%%%%%%%%%%%%%%%%%%%%%%%%%%%%%%%%%%%%%
%\section[References]{References}
%\renewcommand\refname{References}
%\def\newblock{References}
%\begin{frame}[allowframebreaks]{References}
%\bibliography{R:/users/pouliot/Papers/References}
%%\bibliography{D:/Papers/References}
%\end{frame}


%%%%%%%%%%%%%%%%%%%%%%%%%%%%%%%%%%%%%%%%%%%%%%%%%%%%%%%%%%%%%%%%%%%%%%%%%%%%%%%%%%%%%

\end{document}
