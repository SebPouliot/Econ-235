\documentclass[table,xcolor=pdftex,dvipsnames]{beamer}\usepackage[]{graphicx}\usepackage[]{color}
%% maxwidth is the original width if it is less than linewidth
%% otherwise use linewidth (to make sure the graphics do not exceed the margin)
\makeatletter
\def\maxwidth{ %
  \ifdim\Gin@nat@width>\linewidth
    \linewidth
  \else
    \Gin@nat@width
  \fi
}
\makeatother

\definecolor{fgcolor}{rgb}{0.345, 0.345, 0.345}
\newcommand{\hlnum}[1]{\textcolor[rgb]{0.686,0.059,0.569}{#1}}%
\newcommand{\hlstr}[1]{\textcolor[rgb]{0.192,0.494,0.8}{#1}}%
\newcommand{\hlcom}[1]{\textcolor[rgb]{0.678,0.584,0.686}{\textit{#1}}}%
\newcommand{\hlopt}[1]{\textcolor[rgb]{0,0,0}{#1}}%
\newcommand{\hlstd}[1]{\textcolor[rgb]{0.345,0.345,0.345}{#1}}%
\newcommand{\hlkwa}[1]{\textcolor[rgb]{0.161,0.373,0.58}{\textbf{#1}}}%
\newcommand{\hlkwb}[1]{\textcolor[rgb]{0.69,0.353,0.396}{#1}}%
\newcommand{\hlkwc}[1]{\textcolor[rgb]{0.333,0.667,0.333}{#1}}%
\newcommand{\hlkwd}[1]{\textcolor[rgb]{0.737,0.353,0.396}{\textbf{#1}}}%
\let\hlipl\hlkwb

\usepackage{framed}
\makeatletter
\newenvironment{kframe}{%
 \def\at@end@of@kframe{}%
 \ifinner\ifhmode%
  \def\at@end@of@kframe{\end{minipage}}%
  \begin{minipage}{\columnwidth}%
 \fi\fi%
 \def\FrameCommand##1{\hskip\@totalleftmargin \hskip-\fboxsep
 \colorbox{shadecolor}{##1}\hskip-\fboxsep
     % There is no \\@totalrightmargin, so:
     \hskip-\linewidth \hskip-\@totalleftmargin \hskip\columnwidth}%
 \MakeFramed {\advance\hsize-\width
   \@totalleftmargin\z@ \linewidth\hsize
   \@setminipage}}%
 {\par\unskip\endMakeFramed%
 \at@end@of@kframe}
\makeatother

\definecolor{shadecolor}{rgb}{.97, .97, .97}
\definecolor{messagecolor}{rgb}{0, 0, 0}
\definecolor{warningcolor}{rgb}{1, 0, 1}
\definecolor{errorcolor}{rgb}{1, 0, 0}
\newenvironment{knitrout}{}{} % an empty environment to be redefined in TeX

\usepackage{alltt}
%\documentclass[table,xcolor=pdftex,dvipsnames, handout]{beamer}
%\usepackage{handoutWithNotes}
%\pgfpagesuselayout{4 on 1 with notes}[letterpaper,border shrink=5mm] %must also change the document class.


\usepackage{beamerthemesplit}
\usepackage[english]{babel}
\usepackage{amsmath}
\usepackage{amssymb}
\usepackage{amsthm}
\usepackage{verbatim}
\usepackage{graphpap}
\usepackage{epic}
\usepackage{pict2e} %To draw line with any slope
\usepackage{color}
\usepackage{natbib}
\usepackage{enumitem}
\usepackage{booktabs}
\usepackage{xcolor}
\usepackage{textcomp}

\bibliographystyle{ajae}

\newcommand{\p}{\partial}

\newcommand {\framedgraphic}[1] {
        \begin{center}
            \includegraphics[width=\textwidth,height=0.8\textheight,keepaspectratio]{#1}
        \end{center}
        \vspace{-1\baselineskip}
}

\usetheme{Boadilla}
\useoutertheme{shadow}
\usecolortheme{beaver}%seagull
\everymath{\color{blue}}
\everydisplay{\color{blue}}

\usefonttheme{professionalfonts}

\usepackage{hyperref}
\hypersetup{
   colorlinks = {true},
   urlcolor = {blue},
   linkcolor = {black},
   citecolor = {black},
   pdfborderstyle={/S/U/W 1},
   urlbordercolor = 0 0 1,
   citebordercolor = 1 1 1,
   filebordercolor = 1 1 1,
   linkbordercolor = 1 1 1,
   pdfauthor = {Sebastien Pouliot},
}

\widowpenalty=10000 % Avoid single line at the end of a page
\clubpenalty=10000  % Avoid single line at the bottom

\title[Market concentration]{Market concentration in agriculture}
\author[Pouliot]{S\'{e}bastien Pouliot}
\institute{Iowa State University}
\date{Fall 2017}
\IfFileExists{upquote.sty}{\usepackage{upquote}}{}
\begin{document}

%%%%%%%%%%%%%%%%%%%%%%%%%%%%%%%%%%%%%%%%%%%%%%%%%%%%%%%%%%%%%%%%%%%%%%%%%%%%%%%%%%

\begin{frame}
\titlepage
\vspace{-0.4in}
\begin{center}
Lecture notes for Econ 235\\
\end{center}
\end{frame}

%%%%%%%%%%%%%%%%%%%%%%%%%%%%%%%%%%%%%%%%%%%%%%%%%%%%%%%%%%%%%%%%%%%%%%%%%%%%%%%%%%
\section[Introduction]{Introduction}

\begin{frame}{Introduction}
\begin{enumerate}[label=\textbullet]
  \item Recall that the assumptions of perfect competition are:
     \begin{enumerate}[label=\roman{*})]
          \item Firms sell an identical product;
          \item Full information about prices;
          \item The market contains a large number of firms;
          \item There is no transaction cost;
          \item Firms are free to enter and to exit.
      \end{enumerate}
  \item These assumptions do not always hold in agriculture.
  \item In this section, we will look at the consequences of a market with a few buyers and/or a few sellers.
\end{enumerate}
\end{frame}

%%%%%%%%%%%%%%%%%%%%%%%%%%%%%%%%%%%%%%%%%%%%%%%%%%%%%%%%%%%%%%%%%%%%%%%%%%%%%%%%%%
\section[Definitions]{Definitions}

\begin{frame}{Definitions}
\begin{enumerate}[label=\textbullet]
  \item The strength of the assumptions for perfect competition depends greatly on how a market is defined.
  \item When defining a market, one must be careful of geography and product definition.
  \item For example, what is the relevant market when studying concentration of beef packing plants?
       \begin{enumerate}[label=-]
          \item Is it the domestic market for US beef only?
          \item Should it include beef imported from Canada and US exports of beef?
          \item Beef competes with chicken and pork. Thus, should the market be defined with respect to all types of meat?
          \item What about other food products?
      \end{enumerate}
  \item There is no ``correct" way of defining a market.
\end{enumerate}
\end{frame}

%%%%%%%%%%%%%%%%%%%%%%%%%%%%%%%%%%%%%%%%%%%%%%%%%%%%%%%%%%%%%%%%%%%%%%%%%%%%%%%%%%

\begin{frame}{Definitions}
\begin{table}
\scriptsize
\caption{Models of competition}
\begin{tabular}{l l l l l}
  \toprule
  \parbox[b]{0.75in}{\raggedright Characteristics} & \parbox[b]{0.75in}{\raggedright Perfect competition} & \parbox[b]{0.75in}{\raggedright Monopolistic competition} & \parbox[b]{0.75in}{\raggedright Oligopoly (seller) Oligopsony (buyer)} & \parbox[b]{0.75in}{\raggedright Monopoly (seller) Monopsony (buyer)}\\
  \midrule
  \parbox[c]{0.75in}{\raggedright Nature of product} & \parbox[c]{0.75in}{\raggedright Homogeneous} & \parbox[c]{0.75in}{\raggedright Differentiated} & \parbox[c]{0.75in}{\raggedright Homogeneous/ differentiated} & \parbox[c]{0.75in}{\raggedright Differentiated} \\
  \addlinespace[0.075in]
  %\rowcolor{gray!40}
  \parbox[c]{0.75in}{\raggedright Number of firms} & Many & Many & Few & One \\
  \addlinespace[0.075in]
  \rowcolor{white}
  \parbox[c]{0.75in}{\raggedright Ease of entry for new firms} & Easy & Fairly easy & Difficult & \parbox[c]{0.75in}{\raggedright Very difficult to impossible} \\
  \addlinespace[0.075in]
  %\rowcolor{gray!40}
  \parbox[c]{0.75in}{\raggedright Market strategies} & \parbox[c]{0.75in}{\raggedright Timing of sales} & \parbox[c]{0.75in}{\raggedright Set price, brand, names, promotion, product design and packing} & \parbox[c]{0.75in}{\raggedright Set price; if differentiated, then establish brand name, promotion, product design and packaging} & \parbox[c]{0.75in}{\raggedright Set price based on marginal cost equal to marginal revenue}\\
  \bottomrule
\end{tabular}
\end{table}
\end{frame}

%%%%%%%%%%%%%%%%%%%%%%%%%%%%%%%%%%%%%%%%%%%%%%%%%%%%%%%%%%%%%%%%%%%%%%%%%%%%%%%%%%

\begin{frame}{Definitions}
\begin{enumerate}[label=\textbullet]
  \item Monopoly: there is one seller of a product.
  \item Oligopoly: there are a few sellers of a product.
  \item Monopsony: there is one buyer of a product.
  \item Oligopsony: there are a few buyers of a product.
\end{enumerate}
\end{frame}


%%%%%%%%%%%%%%%%%%%%%%%%%%%%%%%%%%%%%%%%%%%%%%%%%%%%%%%%%%%%%%%%%%%%%%%%%%%%%%%%%%
\section{Concentration in agribusinesses}

\begin{frame}{What are agribusiness?}
\begin{enumerate}[label=\textbullet]
  \item Any firm involved in the production, transportation, transformation or sale of food and fiber (e.g. cotton).
  \item Firms that produce inputs for farms are also agribusiness (e.g. production of seeds, fertilizer or machinery).
  \item We can find all types of competitive conduct in agribusiness.
\end{enumerate}
\end{frame}

%%%%%%%%%%%%%%%%%%%%%%%%%%%%%%%%%%%%%%%%%%%%%%%%%%%%%%%%%%%%%%%%%%%%%%%%%%%%%%%%%%

\begin{frame}{Large agribusiness firms}
\begin{enumerate}[label=\textbullet]
  \item The tables below are from Fortune 500 for 2016 available at \url{http://beta.fortune.com/fortune500}.
  \item You can use the filter at the top right corner of the page to select rankings by industries.
\end{enumerate}
\end{frame}

%%%%%%%%%%%%%%%%%%%%%%%%%%%%%%%%%%%%%%%%%%%%%%%%%%%%%%%%%%%%%%%%%%%%%%%%%%%%%%%%%%

\begin{frame}{Large agribusiness firms (2016)}
\begin{table}
\caption{Category: food production}
\begin{tabular}{l c r r}
  \toprule
  Company & \parbox[c]{0.75in}{Fortune 500 rank} & \parbox[c]{0.65in}{\centering Revenues (\$ b)} & \parbox[c]{0.60in}{\centering Profits (\$ M)}\\
  \midrule
  Archer Daniels Midland (ADM) & 45 & 62.3 & 1,279 \\
  Tyson Foods & 82 & 36.9 & 1,768 \\
  CHS & 93 & 30.3 & 424 \\
  Ingredion & 456 & 5.7 & 485 \\
  Seaboard & 486 & 5.4 & 312 \\
  \bottomrule
\end{tabular}
\end{table}
Source: \href{http://fortune.com/fortune500/}{Fortune 500}.
\end{frame}

%%%%%%%%%%%%%%%%%%%%%%%%%%%%%%%%%%%%%%%%%%%%%%%%%%%%%%%%%%%%%%%%%%%%%%%%%%%%%%%%%%

\begin{frame}{Large agribusiness firms (2016)}
\begin{table}
\caption{Category: food services}
\begin{tabular}{l c r r}
  \toprule
  Company & \parbox[c]{0.75in}{Fortune 500 rank} & \parbox[c]{0.65in}{\centering Revenues (\$ b)} & \parbox[c]{0.60in}{\centering Profits (\$ M)}\\
  \midrule
  McDonald's & 112 & 24.6 & 4,687 \\
  Starbucks & 131 & 21.3 & 2,818 \\
  Darden Restaurants & 385 & 6.9 & 375 \\
  Yum China Holdings & 399 & 6.4 & 502 \\
  \bottomrule
\end{tabular}
\end{table}
Source: \href{http://fortune.com/fortune500/}{Fortune 500}.
\end{frame}

%%%%%%%%%%%%%%%%%%%%%%%%%%%%%%%%%%%%%%%%%%%%%%%%%%%%%%%%%%%%%%%%%%%%%%%%%%%%%%%%%%

\begin{frame}{Large agribusiness firms (2016)}
\begin{table}
\caption{Category: beverages}
\begin{tabular}{l c r r}
  \toprule
  Company & \parbox[c]{0.75in}{Fortune 500 rank} & \parbox[c]{0.65in}{\centering Revenues (\$ b)} & \parbox[c]{0.60in}{\centering Profits (\$ M)}\\
  \midrule
  Coca-Cola & 64 & 41.9 & 6,527 \\
  Constellation Brands & 408 & 6.5 & 1,055 \\
  Dr Pepper Snapple Group & 416 & 6.4 & 847 \\
  Molson Coors Brewing & 522 & 4.9 & 1,976 \\
  Coca-Cola Bottling & 701 & 3.1 & 50 \\
  \bottomrule
\end{tabular}
\end{table}
Where is PepsiCo? In the food consumer products category.\\
Source: \href{http://fortune.com/fortune500/}{Fortune 500}.
\end{frame}

%%%%%%%%%%%%%%%%%%%%%%%%%%%%%%%%%%%%%%%%%%%%%%%%%%%%%%%%%%%%%%%%%%%%%%%%%%%%%%%%%%

\begin{frame}{Large agribusiness firms (2016)}
\begin{table}
\caption{Category: food consumer products}
\begin{tabular}{l c r r}
  \toprule
  Company & \parbox[c]{0.75in}{Fortune 500 rank} & \parbox[c]{0.65in}{\centering Revenues (\$ b)} & \parbox[c]{0.60in}{\centering Profits (\$ M)}\\
  \midrule
  PepsiCo & 44 & 62.8 & 6,329 \\
  Kraft Heinz & 106 & 26.5 & 3,632 \\
  Mondel\={e}z International & 109 & 25.9 & 1,659 \\
  General Mills & 165 & 16.6 & 1,697 \\
  ConAgra Brands & 197 & 14.1 & -677 \\
  Land O'Lakes & 209 & 13.2 & 245 \\
  \bottomrule
\end{tabular}
\end{table}
Source: \href{http://fortune.com/fortune500/}{Fortune 500}.
\end{frame}


%%%%%%%%%%%%%%%%%%%%%%%%%%%%%%%%%%%%%%%%%%%%%%%%%%%%%%%%%%%%%%%%%%%%%%%%%%%%%%%%%%
\section{Measuring concentration}

\begin{frame}[allowframebreaks]{Measuring concentration:  concentration ratio}
\begin{enumerate}[label=\textbullet]
  \item Once a market is defined, we may want to measure concentration.
  \item Measure the total revenue of a firm $i$ as $R_i = p q_i$.
  \item The market share of a firm $i$ is equal to its revenue divided by the total revenue of all the firms within a market:\[ s_i = \frac{R_i}{\sum_i^N R_i} \]
  where $N$ is the total number of firms within a market.
  \item Possibly the most common method of measuring concentration in market is to use concentration ratio.
  \item Concentration ratios are the sum of market shares of the $m$ largest firms \[ Cm = \sum_{i=1}^{m} s_i,\].
  \vspace{-\baselineskip}
  \item Often denoted by $C$ or $CR$ followed by the number of firms $m$.
  \item For example, $C4$ is the sum of the market share for the four largest firms in a market.
  \item A small value indicates a low level of concentration.
\end{enumerate}
\end{frame}

%%%%%%%%%%%%%%%%%%%%%%%%%%%%%%%%%%%%%%%%%%%%%%%%%%%%%%%%%%%%%%%%%%%%%%%%%%%%%%%%%%

\begin{frame}{Measuring concentration: Herfindahl index}
\begin{enumerate}[label=\textbullet]
  \item Another measure of concentration is the Herfindahl-Hirschman index (HHI or simply Herfindahl index).
  \item The Herfindahl index is the sum of the square of market shares of all firms \[ HHI = 10,000\sum_{i=1}^{N} s_i^2,\]
  where $N$ is the number of firms.
  \item It is standard practice to multiply by $10,000$ because shares are reported in percentage.
  \item A small value indicates a low level of concentration.
  \item The Herfindahl index adjusts for dispersion in firms' size.
  \item If there are many small firms, then the smallest firms can be ignored without much effect on HHI.
\end{enumerate}
\end{frame}

%%%%%%%%%%%%%%%%%%%%%%%%%%%%%%%%%%%%%%%%%%%%%%%%%%%%%%%%%%%%%%%%%%%%%%%%%%%%%%%%%%

\begin{frame}{Measuring concentration: example (1)}
\begin{enumerate}[label=\textbullet]
  \item Consider the following revenues and market shares:
\end{enumerate}
\begin{table}
\scriptsize
\begin{tabular}{l c c c c c c c c c}
  \toprule
Firm & $A$ & $B$ & $C$ & $D$ & $E$ & $F$ & $G$ & $H$ & Total\\
    \midrule
Revenue & 11,250 & 750 & 750 & 450 & 450 & 450 & 450 & 450 & 15,000\\
\parbox[b][][b]{0.25in}{Market\\share} & 0.75 & 0.05 & 0.05 & 0.03 & 0.03 & 0.03 & 0.03 & 0.03 & 1\\
\parbox[b][][b]{0.25in}{Market\\share$^2$}  & 0.5625 & 0.0025 & 25e-4 & 9e-4 & 9e-4 & 9e-4 & 9e-4 & 9e-4 & 0.5720\\
  \bottomrule
\end{tabular}
\end{table}
\vspace{-0.75\baselineskip}
where 1e-4 = 0.0001.
\begin{enumerate}[label=\textbullet]
  \item Calculate $C4$, $C8$ and the Herfindahl index.
  \pause
  \item $C4=0.88$, $C8=1.00$ and $HHI=5,720$
\end{enumerate}
\end{frame}

%%%%%%%%%%%%%%%%%%%%%%%%%%%%%%%%%%%%%%%%%%%%%%%%%%%%%%%%%%%%%%%%%%%%%%%%%%%%%%%%%%

\begin{frame}{Measuring concentration: example (2)}
\begin{enumerate}[label=\textbullet]
  \item Consider the following revenues and market shares:
\end{enumerate}
\begin{table}
\scriptsize
\begin{tabular}{l c c c c c c c c c}
  \toprule
Firm & $A$ & $B$ & $C$ & $D$ & $E$ & $F$ & $G$ & $H$ & Total\\
    \midrule
Revenue & 3,750 & 3,750 & 3,750 & 1,950 & 450 & 450 & 450 & 450 & 15,000\\
\parbox[b][][b]{0.25in}{Market\\share} & 0.25 & 0.25 & 0.25 & 0.13 & 0.03 & 0.03 & 0.03 & 0.03 & 1\\
\parbox[b][][b]{0.25in}{Market\\share$^2$}  & 0.0625 & 0.0625 & 0.0625 & 0.0169 & 9e-4 & 9e-4 & 9e-4 & 9e-4 & 0.208 \\
  \bottomrule
\end{tabular}
\end{table}
\begin{enumerate}[label=\textbullet]
  \item Calculate $C4$, $C8$ and the Herfindahl index.
  \pause
  \item $C4=0.88$, $C8=1.00$ and $HHI=2,080$
\end{enumerate}
\end{frame}

%%%%%%%%%%%%%%%%%%%%%%%%%%%%%%%%%%%%%%%%%%%%%%%%%%%%%%%%%%%%%%%%%%%%%%%%%%%%%%%%%%

\begin{frame}{Food manufacturing}
\begin{center}
\begin{table}
    \caption{Concentration in food manufacturing (NAICS 311)}
    \begin{tabular}{ l  c  c }%{0.75\textwidth}
      \toprule
       & Concentration ratio & Herfinfal index \\
      \cmidrule(r){2-3}
      4 largest & 16.3 & NA \\
      8 largest & 24.2 & NA \\
      20 largest & 38.4 & NA \\
      50 largest & 50.9 & 110.7 \\
      \bottomrule
    \end{tabular}
\end{table}
\end{center}
Source: \href{http://factfinder.census.gov/faces/tableservices/jsf/pages/productview.xhtml?pid=ECN_2012_US_31SR2&prodType=table}{2012 Economic Census of Manufactures}.
\end{frame}



%%%%%%%%%%%%%%%%%%%%%%%%%%%%%%%%%%%%%%%%%%%%%%%%%%%%%%%%%%%%%%%%%%%%%%%%%%%%%%%%%%

\begin{frame}{Concentration in US grocery}
\begin{center}
\begin{table}
    \caption{Concentration in supermarkets and other grocery (NAICS 445110)}
    \begin{tabular}{ l  c  c }%{0.75\textwidth}
      \toprule
       & Concentration ratio & Herfinfal index \\
      \cmidrule(r){2-3}
      4 largest & 31.1 & NA \\
      8 largest & 44.4 & NA \\
      20 largest & 58.6 & NA \\
      50 largest & 70.6 & NA \\
      \bottomrule
    \end{tabular}
\end{table}
\end{center}
Source: \href{http://factfinder.census.gov/faces/tableservices/jsf/pages/productview.xhtml?pid=ECN_2012_US_44SSSZ6&prodType=table}{2012 Economic Census of Manufactures}.
\end{frame}

%%%%%%%%%%%%%%%%%%%%%%%%%%%%%%%%%%%%%%%%%%%%%%%%%%%%%%%%%%%%%%%%%%%%%%%%%%%%%%%%%%

\begin{frame}{Concentration in meat processed from carcass}
\begin{center}
\begin{table}
    \caption{Concentration in meat processing (NAICS 311612)}
    \begin{tabular}{ l  c  c }%{0.75\textwidth}
      \toprule
       & Concentration ratio & Herfinfal index \\
      \cmidrule(r){2-3}
      4 largest & 32.8 & NA \\
      8 largest & 42.3 & NA \\
      20 largest & 55.4 & NA \\
      50 largest & 71.2 & 332.1 \\
      \bottomrule
    \end{tabular}
\end{table}
\end{center}
Source: \href{http://factfinder.census.gov/faces/tableservices/jsf/pages/productview.xhtml?pid=ECN_2012_US_31SR2&prodType=table}{2012 Economic Census of Manufactures}.
\end{frame}

%%%%%%%%%%%%%%%%%%%%%%%%%%%%%%%%%%%%%%%%%%%%%%%%%%%%%%%%%%%%%%%%%%%%%%%%%%%%%%%%%%
\section{Why more concentration in some sectors?}

\begin{frame}{Why more concentration in some sectors?}
\begin{enumerate}[label=\textbullet]
  \item There is concentration in sectors where there are barriers to entry.
  \item These barriers prevent entry by competing firms, effectively protecting incumbents from new competition.
  \item What are barriers to entry?
      \begin{enumerate}[label=-]
          \item Regulatory (e.g. government controls the number of firms);
          \item Legal (e.g. patent);
          \item Fixed cost (e.g. capital investments such as buildings and land).
      \end{enumerate}
  \item In most sectors, it is the fixed cost that constitutes the largest barrier to entry.
  \item Should observe more concentration in sectors with large fixed costs.
\end{enumerate}
\end{frame}

%%%%%%%%%%%%%%%%%%%%%%%%%%%%%%%%%%%%%%%%%%%%%%%%%%%%%%%%%%%%%%%%%%%%%%%%%%%%%%%%%%
\section{Cartels in agriculture}

\begin{frame}{Cooperatives in agriculture}
\begin{enumerate}[label=\textbullet]
  \item In the United States, the \emph{Sherman Antitrust Act of 1890} (Sherman Act) makes illegal business practices that reduce competition.
  \item Monopolies and cartels are illegal in the United States.
  \item For example, see the recent antitrust case with eggs (\url{http://www.eggproductssettlement.com/}).
  \item Another example is recent case about canned tuna (e.g. \href{http://talkbusiness.net/2016/11/wal-mart-the-latest-in-a-growing-number-to-file-an-antitrust-suit-against-canned-tuna-suppliers/}{Wal-Mart joins lawsuit}).
  \item The \emph{Capper-Volstead Act of 1922} provides agricultural producers certain exemption from antitrust laws.
  \item Capper-Volstead allows for the formation of cooperatives under specific rules (e.g. one vote per member).
  \item Cooperatives provide a way for farmers to join their force to increase their market power.
\end{enumerate}
\end{frame}

%%%%%%%%%%%%%%%%%%%%%%%%%%%%%%%%%%%%%%%%%%%%%%%%%%%%%%%%%%%%%%%%%%%%%%%%%%%%%%%%%%

\begin{frame}{Federal marketing orders}
\begin{enumerate}[label=\textbullet]
  \item Marketing orders and marketing agreements are designed to stabilize market conditions for certain agricultural commodities. (Consider this a claim rather than an actual fact.)
  \item A marketing order covers all firms within an industry (e.g. Milk marketing order - Got milk?).
  \item A marketing agreements cover participating firms within an industry (e.g. California Leafy-Green Marketing Agreement).
  \item A few marketing orders limit production quantity, effectively giving market power to producers through the creation of a cartel (milk, fruits, vegetables and nuts).
  \item In Canada, production quotas in dairy, chicken, eggs and maple syrup (Quebec only) effectively give producers market power.
\end{enumerate}
\end{frame}

%%%%%%%%%%%%%%%%%%%%%%%%%%%%%%%%%%%%%%%%%%%%%%%%%%%%%%%%%%%%%%%%%%%%%%%%%%%%%%%%%%

\section{Consequences of Market power}

\begin{frame}{Concentration and market power}
\begin{enumerate}[label=\textbullet]
  \item Concentration is a signal that firms may be exercising market power.
  \item Concentration is however not a sufficient condition for market power.
  \item Threats of entry and regulation may prevent firms from excising market power.
  \item That is, a firm might be in a position to exercise market power but chooses not to by keeping its prices low to prevent entry by competing firms.
\end{enumerate}
\end{frame}

%%%%%%%%%%%%%%%%%%%%%%%%%%%%%%%%%%%%%%%%%%%%%%%%%%%%%%%%%%%%%%%%%%%%%%%%%%%%%%%%%%

\begin{frame}{What are the consequences of market power by sellers?}
\begin{enumerate}[label=\textbullet]
  \item For competitive sellers, the price equals the marginal cost ($p=mc$).
  \item Sellers with market power choose output quantities such that marginal revenue equals marginal cost ($mr=mc$) and the price is given along the demand curve. This yields an output price that is greater than the marginal cost ($p>mc$).
  \item Market power by sellers causes a decline in the quantity and an increase in the prices compared to the case with competitive firms.
  \item Market power causes a welfare loss compare to the case with perfect competition.
  \item Transfer of surplus from buyers to the sellers.
  \item The deadweight loss triangle (dwl) is a loss of surplus (welfare) to society.
\end{enumerate}
\end{frame}

%%%%%%%%%%%%%%%%%%%%%%%%%%%%%%%%%%%%%%%%%%%%%%%%%%%%%%%%%%%%%%%%%%%%%%%%%%%%%%%%%%

\begin{frame}{Market power by a seller}
\begin{figure}[htbp]
\begin{center}
    \begin{picture}(240,180)
        %Axises and labels
        \scriptsize
        \put(0,0){\vector(1,0){240}} %x-axis
        \put(0,0){\vector(0,1){180}} %y-axis
        \put(225,-10){$Q$}
        \put(-5,170){\makebox(0,0){$P$}}
        %Demand curve
        \thicklines
        \put(0,160){\line(1,-1){160}}
        %Supply curve
        \put(0,0){\line(1,1){160}}
        %Text
        \put(155,10){$D$}
        \put(140,155){$mc$}
        %Equilibrium
        \color{black}
        \multiput(0,80)(5,0){16}{\line(1,0){2.5}}%Dashed line
        \multiput(80,80)(0,-5){16}{\line(0,-1){2.5}}%Dashed line
        \put(-10,78){$P^\ast$}
        \put(78,-10){$Q^\ast$}
        %Market power by seller
        \onslide<2->
        \color{blue}
        %Demand curve
        \thicklines
        \put(0,160){\line(1,-2){80}}
        \put(65,30){$mr$}
        \onslide<3->
        \multiput(0,106.67)(5,0){11}{\line(1,0){2.5}}%Dashed line h
        \multiput(53.33,106.67)(0,-5){22}{\line(0,-1){2.5}}%Dashed line v
        \put(47,-10){$Q_M$}
        \put(-15,105){$P_M$}
        \onslide<4>
        \put(55,85){$dwl$}
        \put(55,70){$dwl$}
    \end{picture}
\vspace{0.1in}
\caption{Market power by a seller} \label{fig.seller}
\end{center}
\end{figure}
\end{frame}

%%%%%%%%%%%%%%%%%%%%%%%%%%%%%%%%%%%%%%%%%%%%%%%%%%%%%%%%%%%%%%%%%%%%%%%%%%%%%%%%%%

\begin{frame}{What are the consequences of market power by buyers?}
\begin{enumerate}[label=\textbullet]
  \item The consequences are very similar to the case with market power by sellers
  \item For competitive buyers, the price of an input $w$ equals the marginal marginal revenue product ($w=mrp$), which is the demand for that input.
  \item The marginal revenue product is how much an additional unit of product increases revenue.
  \item A buyer with market power chooses input quantities such that its marginal expenditure equals its marginal revenue product ($me=mrp$) and the price is given along the supply curve. This yields an input price that is smaller than the marginal value product ($w<mrp$).
  \item Market power by buyers causes a decline in the quantity and a decline in the prices of input compared to the case with competitive firms.
  \item Transfer of surplus from sellers to buyers.
  \item Welfare deadweight loss triangle (dwl).
\end{enumerate}
\end{frame}

%%%%%%%%%%%%%%%%%%%%%%%%%%%%%%%%%%%%%%%%%%%%%%%%%%%%%%%%%%%%%%%%%%%%%%%%%%%%%%%%%%

\begin{frame}{Market power by a buyer}
\begin{figure}[htbp]
\begin{center}
    \begin{picture}(240,180)
        %Axises and labels
        \scriptsize
        \put(0,0){\vector(1,0){240}} %x-axis
        \put(0,0){\vector(0,1){180}} %y-axis
        \put(225,-10){$X$}
        \put(-5,170){\makebox(0,0){$W$}}
        %Demand curve
        \thicklines
        \put(0,160){\line(1,-1){160}}
        %Supply curve
        \put(0,0){\line(1,1){160}}
        %Text
        \put(155,10){$mrp$}
        \put(150,155){$S$}
        %Equilibrium
        \color{black}
        \multiput(0,80)(5,0){16}{\line(1,0){2.5}}%Dashed line
        \multiput(80,80)(0,-5){16}{\line(0,-1){2.5}}%Dashed line
        \put(-12,78){$W^\ast$}
        \put(78,-10){$X^\ast$}
        %Market power by seller
        \onslide<2->
        \color{blue}
        %Demand curve
        \thicklines
        \put(0,0){\line(1,2){80}}
        \put(65,155){$me$}
        \onslide<3->
        \multiput(0,53.33)(5,0){11}{\line(1,0){2.5}}%Dashed line h
        \multiput(53.33,106.67)(0,-5){22}{\line(0,-1){2.5}}%Dashed line v
        \put(47,-10){$X_M$}
        \put(-15,50){$W_M$}
        \onslide<4>
        \put(55,85){$dwl$}
        \put(55,70){$dwl$}
    \end{picture}
\vspace{0.1in}
\caption{Market power by a buyer} \label{fig.buyer}
\end{center}
\end{figure}
\end{frame}


%%%%%%%%%%%%%%%%%%%%%%%%%%%%%%%%%%%%%%%%%%%%%%%%%%%%%%%%%%%%%%%%%%%%%%%%%%%%%%%%%%

\begin{frame}{Economies of scales and scopes}
\begin{enumerate}[label=\textbullet]
  \item Large fixed costs are associated with economies of scales.
  \item \emph{Economies of scales} mean that the average cost of a product declines as the quantity of product increases.
  \item This means that when there are economies of scales, larger firms can sell a product at a lower price.
  \item \emph{Economies of scopes} arise when a firm can produce many products at a lower cost than many firms.
  \item For example, a grocery store can sell thousand of different products at a lower cost than thousand stores each selling one product.
\end{enumerate}
\end{frame}

%%%%%%%%%%%%%%%%%%%%%%%%%%%%%%%%%%%%%%%%%%%%%%%%%%%%%%%%%%%%%%%%%%%%%%%%%%%%%%%%%%

\begin{frame}{Are concentration and market power by agribusiness firms big problems?}
\begin{enumerate}[label=\textbullet]
  \item Farm sector sometimes complain about market power by buyers:
      \begin{enumerate}[label=-]
          \item Vertical integration in poultry - see NPR stories \href{http://www.npr.org/blogs/thesalt/2014/02/20/279040721/the-system-that-supplies-our-chickens-pits-farmer-against-farmer}{here} and \href{http://www.npr.org/blogs/thesalt/2014/02/19/276981085/is-tyson-foods-chicken-empire-a-meat-racket}{here};
          \item Vertical integration is also more and more common in hogs;
          \item Several litigation cases in cattle.
      \end{enumerate}
      \item At retail, concerns about increased concentration with the growth of large surface stores:
      \begin{enumerate}[label=-]
          \item Fear that Walmart would increase prices after driving out the small stores never materialized;
          \item Large retailers tend to exercise market power more as buyers than sellers.
      \end{enumerate}
\end{enumerate}
\end{frame}

%%%%%%%%%%%%%%%%%%%%%%%%%%%%%%%%%%%%%%%%%%%%%%%%%%%%%%%%%%%%%%%%%%%%%%%%%%%%%%%%%%

\begin{frame}{Are concentration and market power by agribusiness firms big problems?}
\begin{enumerate}[label=\textbullet]
  \item By itself, concentration is not a problem.
  \item The problem is firms exercising market power because it causes a loss in welfare.
  \item If market power is associated with economies of scales or scopes, then market power might yield a larger total welfare than perfect competition.
  \item Most agribusiness firms are unable to exercise significant market power:
      \begin{enumerate}[label=-]
          \item Threat of entry by competing firms;
          \item Threat of antitrust litigation.
      \end{enumerate}
  \item Thus, concentration and market power are not such a big deal in agribusiness.
\end{enumerate}
\end{frame}


%%%%%%%%%%%%%%%%%%%%%%%%%%%%%%%%%%%%%%%%%%%%%%%%%%%%%%%%%%%%%%%%%%%%%%%%%%%%%%%%%%

\begin{frame}[allowframebreaks]{Market power: what to do?}
\begin{enumerate}[label=\textbullet]
  \item  In the United States, the Sherman Antitrust Act of 1890 has been quite effective at protecting consumers against ``abusive" exercise of market power.
  \item Other steps to favor competitions include mechanisms to relay information about prices, which facilitates arbitrage between market and thus contribute to limit the exercise of market power:
    \begin{enumerate}[label=-]
        \item The USDA through several of its agencies (e.g. ERS, NASS, FAS) report prices for agricultural commodities.
        \item The US government requires mandatory price reporting for several commodities (see for example the \emph{Mandatory Price Reporting Act} of 2010). The objective is to improve transparency and favor competition.
  \end{enumerate}
  \item International trade increases competitions within a market.
  \item However, the Trump administration recently removed an interim \href{http://www.beefmagazine.com/ranching/usda-dumps-controversial-gipsa-rule}{GIPSA} rule about fair pricing in the livestock industry. 
\end{enumerate}
\end{frame}



%%%%%%%%%%%%%%%%%%%%%%%%%%%%%%%%%%%%%%%%%%%%%%%%%%%%%%%%%%%%%%%%%%%%%%%%%%%%%%%%%%%%%
\end{document}

