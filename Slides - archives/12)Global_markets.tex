\documentclass[table,xcolor=pdftex,dvipsnames]{beamer}\usepackage[]{graphicx}\usepackage[]{color}
%% maxwidth is the original width if it is less than linewidth
%% otherwise use linewidth (to make sure the graphics do not exceed the margin)
\makeatletter
\def\maxwidth{ %
  \ifdim\Gin@nat@width>\linewidth
    \linewidth
  \else
    \Gin@nat@width
  \fi
}
\makeatother

\definecolor{fgcolor}{rgb}{0.345, 0.345, 0.345}
\newcommand{\hlnum}[1]{\textcolor[rgb]{0.686,0.059,0.569}{#1}}%
\newcommand{\hlstr}[1]{\textcolor[rgb]{0.192,0.494,0.8}{#1}}%
\newcommand{\hlcom}[1]{\textcolor[rgb]{0.678,0.584,0.686}{\textit{#1}}}%
\newcommand{\hlopt}[1]{\textcolor[rgb]{0,0,0}{#1}}%
\newcommand{\hlstd}[1]{\textcolor[rgb]{0.345,0.345,0.345}{#1}}%
\newcommand{\hlkwa}[1]{\textcolor[rgb]{0.161,0.373,0.58}{\textbf{#1}}}%
\newcommand{\hlkwb}[1]{\textcolor[rgb]{0.69,0.353,0.396}{#1}}%
\newcommand{\hlkwc}[1]{\textcolor[rgb]{0.333,0.667,0.333}{#1}}%
\newcommand{\hlkwd}[1]{\textcolor[rgb]{0.737,0.353,0.396}{\textbf{#1}}}%
\let\hlipl\hlkwb

\usepackage{framed}
\makeatletter
\newenvironment{kframe}{%
 \def\at@end@of@kframe{}%
 \ifinner\ifhmode%
  \def\at@end@of@kframe{\end{minipage}}%
  \begin{minipage}{\columnwidth}%
 \fi\fi%
 \def\FrameCommand##1{\hskip\@totalleftmargin \hskip-\fboxsep
 \colorbox{shadecolor}{##1}\hskip-\fboxsep
     % There is no \\@totalrightmargin, so:
     \hskip-\linewidth \hskip-\@totalleftmargin \hskip\columnwidth}%
 \MakeFramed {\advance\hsize-\width
   \@totalleftmargin\z@ \linewidth\hsize
   \@setminipage}}%
 {\par\unskip\endMakeFramed%
 \at@end@of@kframe}
\makeatother

\definecolor{shadecolor}{rgb}{.97, .97, .97}
\definecolor{messagecolor}{rgb}{0, 0, 0}
\definecolor{warningcolor}{rgb}{1, 0, 1}
\definecolor{errorcolor}{rgb}{1, 0, 0}
\newenvironment{knitrout}{}{} % an empty environment to be redefined in TeX

\usepackage{alltt}
%\documentclass[table,xcolor=pdftex,dvipsnames, handout]{beamer}

%\usepackage{handoutWithNotes}
%\pgfpagesuselayout{4 on 1 with notes}[letterpaper,border shrink=5mm]

\usepackage{beamerthemesplit}
\usepackage[english]{babel}
\usepackage{amsmath}
\usepackage{amssymb}
\usepackage{amsthm}
\usepackage{verbatim}
\usepackage{graphpap}
\usepackage{epic}
\usepackage{pict2e} %To draw line with any slope
\usepackage{color}
\usepackage{natbib}
\usepackage{enumitem}
\usepackage{booktabs}
\usepackage{xcolor}

\bibliographystyle{ajae}

\newcommand{\p}{\partial}

\newcommand {\framedgraphic}[1] {
        \begin{center}
            \includegraphics[width=\textwidth,height=0.7\textheight,keepaspectratio]{#1}
        \end{center}
        \vspace{-1\baselineskip}
}

\usetheme{Boadilla}
\useoutertheme{shadow}
\usecolortheme{beaver}%seagull
\everymath{\color{blue}}
\everydisplay{\color{blue}}

\usefonttheme{serif}

\usepackage{hyperref}
\hypersetup{
   colorlinks = {true},
   urlcolor = {blue},
   linkcolor = {black},
   citecolor = {black},
   pdfborderstyle={/S/U/W 1},
   urlbordercolor = 0 0 1,
   citebordercolor = 1 1 1,
   filebordercolor = 1 1 1,
   linkbordercolor = 1 1 1,
   pdfauthor = {Sebastien Pouliot}
}

\widowpenalty=10000 % Avoid single line at the end of a page
\clubpenalty=10000  % Avoid single line at the bottom

\title[Agricultural trade]{Agricultural trade}
\author[Pouliot]{S\'{e}bastien Pouliot}
\institute{Iowa State University}
\date{Fall 2017}
\IfFileExists{upquote.sty}{\usepackage{upquote}}{}
\begin{document}

%%%%%%%%%%%%%%%%%%%%%%%%%%%%%%%%%%%%%%%%%%%%%%%%%%%%%%%%%%%%%%%%%%%%%%%%%%%%%%%%%%

\begin{frame}
\titlepage
\vspace{-0.4in}
\begin{center}
Lecture notes for Econ 235\\
\end{center}
\end{frame}

%%%%%%%%%%%%%%%%%%%%%%%%%%%%%%%%%%%%%%%%%%%%%%%%%%%%%%%%%%%%%%%%%%%%%%%%%%%%%%%%%%
\section[Introduction]{Introduction}

\begin{frame}{International trade in agriculture}
\begin{enumerate}[label=\textbullet]
  \item International trade has a very significant impact on US agriculture.
  \item For many commodities, prices are determined on the international market.
  \item International agreements about rules of trade have shaped farm policies in the United States and other countries.
  \item For firms, entering the international market is costly and risky.
\end{enumerate}
\end{frame}

%%%%%%%%%%%%%%%%%%%%%%%%%%%%%%%%%%%%%%%%%%%%%%%%%%%%%%%%%%%%%%%%%%%%%%%%%%%%%%%%%%

\begin{frame}{To read}
\begin{enumerate}[label=\textbullet]
    \item For a description of US agricultural trade: \href{http://www.fas.org/sgp/crs/misc/98-253.pdf}{U.S. Agricultural Trade: Trends, Composition, Direction, and Policy.}
    \item For a brief description of the WTO agreement on agriculture: \href{http://www.wto.org/english/docs_e/legal_e/ursum_e.htm}{Summary of WTO agreement on agriculture.}
    \item If you want access data about US agricultural trade, look at the \href{http://www.fas.usda.gov/psdonline/psdHome.aspx}{USDA Foreign Agricultural Service website}.
\end{enumerate}
\end{frame}

%%%%%%%%%%%%%%%%%%%%%%%%%%%%%%%%%%%%%%%%%%%%%%%%%%%%%%%%%%%%%%%%%%%%%%%%%%%%%%%%%%

\begin{frame}{Monthly US agricultural trade volumes}
\begin{knitrout}
\definecolor{shadecolor}{rgb}{0.969, 0.969, 0.969}\color{fgcolor}\begin{kframe}


{\ttfamily\noindent\bfseries\color{errorcolor}{\#\# Error in filter\_impl(.data, quo): Evaluation error: object 'X2' not found.}}

{\ttfamily\noindent\bfseries\color{errorcolor}{\#\# Error in print(agtrade): object 'agtrade' not found}}\end{kframe}
\end{knitrout}
\scriptsize
Source: The data are from \href{https://www.ers.usda.gov/data-products/foreign-agricultural-trade-of-the-united-states-fatus/}{FATUS}.
\end{frame}

%%%%%%%%%%%%%%%%%%%%%%%%%%%%%%%%%%%%%%%%%%%%%%%%%%%%%%%%%%%%%%%%%%%%%%%%%%%%%%%%%%

\begin{frame}{Monthly US agricultural trade balance}
\begin{knitrout}
\definecolor{shadecolor}{rgb}{0.969, 0.969, 0.969}\color{fgcolor}\begin{kframe}


{\ttfamily\noindent\bfseries\color{errorcolor}{\#\# Error in print(balancetrade): object 'balancetrade' not found}}\end{kframe}
\end{knitrout}
\scriptsize
Source: The data are from \href{https://www.ers.usda.gov/data-products/foreign-agricultural-trade-of-the-united-states-fatus/}{FATUS}.
\end{frame}

%%%%%%%%%%%%%%%%%%%%%%%%%%%%%%%%%%%%%%%%%%%%%%%%%%%%%%%%%%%%%%%%%%%%%%%%%%%%%%%%%%

\begin{frame}{Top 15 US agricultural exports by country in 2015}
\begin{knitrout}
\definecolor{shadecolor}{rgb}{0.969, 0.969, 0.969}\color{fgcolor}\begin{kframe}


{\ttfamily\noindent\bfseries\color{errorcolor}{\#\# Error in print(Exportsbycountry): object 'Exportsbycountry' not found}}\end{kframe}
\end{knitrout}
\scriptsize
Source: The data are from \href{https://www.ers.usda.gov/data-products/foreign-agricultural-trade-of-the-united-states-fatus/}{FATUS}.
\end{frame}

%%%%%%%%%%%%%%%%%%%%%%%%%%%%%%%%%%%%%%%%%%%%%%%%%%%%%%%%%%%%%%%%%%%%%%%%%%%%%%%%%%

\begin{frame}{Top 15 US agricultural imports by country in 2015}
\begin{knitrout}
\definecolor{shadecolor}{rgb}{0.969, 0.969, 0.969}\color{fgcolor}\begin{kframe}


{\ttfamily\noindent\bfseries\color{errorcolor}{\#\# Error in print(Importsbycountry): object 'Importsbycountry' not found}}\end{kframe}
\end{knitrout}
\scriptsize
Source: The data are from \href{https://www.ers.usda.gov/data-products/foreign-agricultural-trade-of-the-united-states-fatus/}{FATUS}.
\end{frame}

%%%%%%%%%%%%%%%%%%%%%%%%%%%%%%%%%%%%%%%%%%%%%%%%%%%%%%%%%%%%%%%%%%%%%%%%%%%%%%%%%%

\begin{frame}{Top 15 US agricultural exports by commodities in 2015}
\begin{knitrout}
\definecolor{shadecolor}{rgb}{0.969, 0.969, 0.969}\color{fgcolor}\begin{kframe}


{\ttfamily\noindent\bfseries\color{errorcolor}{\#\# Error in print(Exportsbycommodities): object 'Exportsbycommodities' not found}}\end{kframe}
\end{knitrout}
\scriptsize
Source: The data are from \href{https://www.ers.usda.gov/data-products/foreign-agricultural-trade-of-the-united-states-fatus/}{FATUS}.
\end{frame}

%%%%%%%%%%%%%%%%%%%%%%%%%%%%%%%%%%%%%%%%%%%%%%%%%%%%%%%%%%%%%%%%%%%%%%%%%%%%%%%%%%

\begin{frame}{Top 15 US agricultural imports by commodities in 2015}
\begin{knitrout}
\definecolor{shadecolor}{rgb}{0.969, 0.969, 0.969}\color{fgcolor}\begin{kframe}


{\ttfamily\noindent\bfseries\color{errorcolor}{\#\# Error in print(Importsbycommodities): object 'Importsbycommodities' not found}}\end{kframe}
\end{knitrout}
\scriptsize
Source: The data are from \href{https://www.ers.usda.gov/data-products/foreign-agricultural-trade-of-the-united-states-fatus/}{FATUS}.
\end{frame}

%%%%%%%%%%%%%%%%%%%%%%%%%%%%%%%%%%%%%%%%%%%%%%%%%%%%%%%%%%%%%%%%%%%%%%%%%%%%%%%%%%
\section{Comparative advantage}

\begin{frame}{Definitions}
\begin{enumerate}[label=\textbullet]
  \item Why do firms across countries trade? Profits!
  \item It is common in economics to refer as countries trading commodities while in fact firms are the one trading. To simplify the discussion, suppose that countries trade between each other.
  \item One common explanation for trade is the existence of \emph{comparative advantages}.
  \item In the context of international trade, comparative advantage is the concept that a country is relatively more efficient at producing a good compared to another country.
  \item The concept of \emph{absolute advantage} is that a country is more efficient at producing all goods than another countries.
\end{enumerate}
\end{frame}


%%%%%%%%%%%%%%%%%%%%%%%%%%%%%%%%%%%%%%%%%%%%%%%%%%%%%%%%%%%%%%%%%%%%%%%%%%%%%%%%%%

\begin{frame}{Comparative advantage vs. absolute advantage}
\begin{enumerate}[label=\textbullet]
  \item Suppose that there is two countries, two goods and one factor of production:
      \begin{enumerate}[label=-]
          \item Country A and country B;
          \item Lettuce and spinach;
          \item The only factor of production is labor and each country has a quantity of labor equal to 1.
      \end{enumerate}
  \item For country A:
      \begin{enumerate}[label=-]
          \item If it invests all its labor in the production of spinach, it can produce 5 units of spinach;
          \item If it invests all its labor in the production of lettuce, it can produce 4 units of lettuce;
          \item It can allocate a fraction of its labor to spinach and lettuce. However, for each unit of lettuce that it produces, it must sacrifice $\frac{5}{4}$ unit of spinach.
          \item $\frac{5}{4}$ is the marginal rate of transformation (MRT) between spinach and lettuce in country A.
      \end{enumerate}

\end{enumerate}
\end{frame}

%%%%%%%%%%%%%%%%%%%%%%%%%%%%%%%%%%%%%%%%%%%%%%%%%%%%%%%%%%%%%%%%%%%%%%%%%%%%%%%%%%

\begin{frame}{Comparative advantage vs. absolute advantage}
\begin{enumerate}[label=\textbullet]
  \item For country B:
      \begin{enumerate}[label=-]
          \item If it invests all its labor in the production of spinach, it can produce 2 units of spinach;
          \item If it invests all its labor in the production of lettuce, it can produce 4 units of lettuce;
          \item It can allocate a fraction of it labor to spinach and lettuce. However, for each unit of lettuce that it produces, it must sacrifice $\frac{2}{4} = \frac{1}{2}$ unit of spinach.
          \item $\frac{1}{2}$ is the marginal rate of transformation (MRT) between spinach and lettuce in country A.
      \end{enumerate}
\end{enumerate}
\end{frame}

%%%%%%%%%%%%%%%%%%%%%%%%%%%%%%%%%%%%%%%%%%%%%%%%%%%%%%%%%%%%%%%%%%%%%%%%%%%%%%%%%%

\begin{frame}{Absolute advantage}
\begin{enumerate}[label=\textbullet]
  \item Country A has an absolute advantage because it is at least as efficient at producing lettuce and spinach than country B:
      \begin{enumerate}[label=-]
          \item With one unit of labor, country A can produce 5 units of spinach while country B can produce only 2 units of spinach.
          \item With one unit of labor, country A can produce the same quantity of lettuce (4 units) as country B.
      \end{enumerate}
\end{enumerate}
\end{frame}

%%%%%%%%%%%%%%%%%%%%%%%%%%%%%%%%%%%%%%%%%%%%%%%%%%%%%%%%%%%%%%%%%%%%%%%%%%%%%%%%%%

\begin{frame}{Comparative advantage}
\begin{enumerate}[label=\textbullet]
    \item Country A has a comparative advantage in the production of spinach:
      \begin{enumerate}[label=-]
        \item For each unit of spinach that it produces, country A must sacrifice $\frac{4}{5}$ unit of lettuce.
        \item For each unit of spinach that it produces, country B must sacrifice $2$ units of lettuce.
      \end{enumerate}
    \item Country B has a comparative advantage in the production of lettuce:
      \begin{enumerate}[label=-]
        \item For each unit of lettuce that it produces, country A must sacrifice $\frac{5}{4}$ unit of spinach.
        \item For each unit of lettuce that it produces, country B must sacrifice $\frac{1}{2}$ unit of spinach.
      \end{enumerate}
    \item You can think of the relative quantities as prices.
\end{enumerate}
\end{frame}

%%%%%%%%%%%%%%%%%%%%%%%%%%%%%%%%%%%%%%%%%%%%%%%%%%%%%%%%%%%%%%%%%%%%%%%%%%%%%%%%%%

\begin{frame}{Comparative advantage and trade}
\begin{enumerate}[label=\textbullet]
    \item Comparative advantages explain trade.
    \item Absolute advantage does not explain trade.
    \item Suppose that country A is in \emph{autarky} (no trade) and that it allocates half of its labor to spinach and half of its labor to lettuce:
      \begin{enumerate}[label=-]
        \item Country A produces 2.5 units of spinach and 2 units of lettuce.
      \end{enumerate}
    \item Suppose that country B is in autarky (no trade) and that it allocates half of its labor to spinach and half of its labor to lettuce:
      \begin{enumerate}[label=-]
        \item Country B produces 1 units of spinach and 2 units of lettuce.
      \end{enumerate}
\end{enumerate}
\end{frame}

%%%%%%%%%%%%%%%%%%%%%%%%%%%%%%%%%%%%%%%%%%%%%%%%%%%%%%%%%%%%%%%%%%%%%%%%%%%%%%%%%%

\begin{frame}{Comparative advantage and trade}
\begin{enumerate}[label=\textbullet]
    \item Suppose instead that the two countries enter into trade, specialize in the product where they have a comparative advantage and then split the total production.
    \item Country A has a comparative advantage in the production of spinach and thus specializes in the production of spinach; it produces 5 units of spinach.
    \item Country B has a comparative advantage in the production of lettuce and thus specializes in the production of spinach; it produces 4 units of lettuce.
    \item Then, one possible allocation of the two goods is:
      \begin{enumerate}[label=-]
        \item Country A receives 3 units of spinach and 2 units of lettuce.
        \item Country B receives 2 units of spinach and 2 units of lettuce.
      \end{enumerate}
    \item Both countries are better off.
\end{enumerate}
\end{frame}

%%%%%%%%%%%%%%%%%%%%%%%%%%%%%%%%%%%%%%%%%%%%%%%%%%%%%%%%%%%%%%%%%%%%%%%%%%%%%%%%%%

\begin{frame}{Comparative advantage and trade}
\begin{enumerate}[label=\textbullet]
    \item Thus, by allowing free trade, countries can specialize and increase total production.
    \item Both countries improve their situation.
    \item In practice, the allocation of goods across countries is done through equilibrium prices. We will not cover this type of solution here.
    \item Note that comparative advantages can be seasonal (e.g. fresh fruits and vegetables), which gives rise to two-way trade.
\end{enumerate}
\end{frame}

%%%%%%%%%%%%%%%%%%%%%%%%%%%%%%%%%%%%%%%%%%%%%%%%%%%%%%%%%%%%%%%%%%%%%%%%%%%%%%%%%%
\section{Free trade}

\begin{frame}{Free trade in practice}
\begin{enumerate}[label=\textbullet]
    \item In practice, free trade almost never occurs even though, as the previous example shows, free trade is welfare improving.
    \item This is particularly true in agriculture, a sector for which countries tend to be more protective.
    \item Is agriculture a special case?
\end{enumerate}
\end{frame}

%%%%%%%%%%%%%%%%%%%%%%%%%%%%%%%%%%%%%%%%%%%%%%%%%%%%%%%%%%%%%%%%%%%%%%%%%%%%%%%%%%

\begin{frame}{Motives for trade barriers}
\begin{enumerate}[label=\textbullet]
    \item Many motives to limit trade:
    \begin{enumerate}[label=\roman*)]
        \item Tax revenues.
        \item Food security - assuring sufficient production of food domestically.
        \item Protect domestic firms (and workers), pressure from domestic interest groups (rent seeking).
        \item Sanitary and phytosanitary (animal and plant diseases, pests and diseases) concerns.
        \item Retaliation in trade disputes.
        \item Prevent dumping.
        \item Countervailing duties.
    \end{enumerate}
    \item From an economic point of view, policies that limit trade are bad. Trade restrictions usually involve a transfer to a group at the expense of another.
\end{enumerate}
\end{frame}

%%%%%%%%%%%%%%%%%%%%%%%%%%%%%%%%%%%%%%%%%%%%%%%%%%%%%%%%%%%%%%%%%%%%%%%%%%%%%%%%%%

\begin{frame}{Efforts toward freeing trade}
\begin{enumerate}[label=\textbullet]
    \item In the last few decades, many countries have entered into negotiations with the objective of opening their borders.
    \item The Generalized Agreement on Tariffs and Trade (GATT) was ratified by many developed countries in 1948 and remained effective until 1994. The agreement provided rules for \href{http://www.wto.org/english/thewto_e/whatis_e/tif_e/fact4_e.htm}{international trade}.
    \item The two basic principles of the GATT are:
      \begin{enumerate}[label=-]
        \item \textbf{Most-Favored-Nation}: Countries cannot discriminate between their trading partners. Exemptions for regional trade agreements and preferential treatments of developing countries.
        \item \textbf{National treatment}: Imported and locally-produced goods should be treated equally.
      \end{enumerate}
\end{enumerate}
\end{frame}

%%%%%%%%%%%%%%%%%%%%%%%%%%%%%%%%%%%%%%%%%%%%%%%%%%%%%%%%%%%%%%%%%%%%%%%%%%%%%%%%%%

\begin{frame}{Efforts toward freeing trade}
\begin{enumerate}[label=\textbullet]
    \item Many attempts, through several rounds of negotiations, were made to update and expand the GATT agreement.
    \item The Uruguay Round of negotiations concluded in 1994 with the creation of the World Trade Organization (WTO).
      \begin{enumerate}[label=-]
        \item The agreement became effective in 1995;
        \item GATT became one of the agreement under the WTO, covering the trade of goods;
        \item GATS is the General Agreement on Trade and Services;
        \item TRIPS is the agreement on Trade-Related Aspects of Intellectual Property Rights
      \end{enumerate}
\end{enumerate}
\end{frame}

%%%%%%%%%%%%%%%%%%%%%%%%%%%%%%%%%%%%%%%%%%%%%%%%%%%%%%%%%%%%%%%%%%%%%%%%%%%%%%%%%%
\section{The World Trade Organization}

\begin{frame}{WTO and agriculture}
\begin{enumerate}[label=\textbullet]
    \item In the Uruguay Round of negotiations, agriculture was a major road block:
      \begin{enumerate}[label=-]
        \item Difficulty in agreeing on discipline regarding subsidies to agriculture;
        \item Difficulty in agreeing on the reduction of trade barriers.
        \item This contrasts with manufacturing products that receive little protection and are generally openly traded.
      \end{enumerate}
    \item In the WTO Agreement on Agriculture (AoA), countries agreed to improve market access and reduce trade-distorting domestic support.
\end{enumerate}
\end{frame}

%%%%%%%%%%%%%%%%%%%%%%%%%%%%%%%%%%%%%%%%%%%%%%%%%%%%%%%%%%%%%%%%%%%%%%%%%%%%%%%%%%

\begin{frame}[allowframebreaks]{WTO and support to agriculture}
\begin{enumerate}[label=\textbullet]
    \item The AoA called for the reduction of support to agriculture that distort production:
      \begin{enumerate}[label=-]
        \item ``Over-production'' lowers international prices for agricultural commodities, most negatively affecting developing countries that cannot afford to subsidize their agriculture.
        \item This meant the end of agricultural support that is tied to the price.
      \end{enumerate}
  \framebreak
   \item Under AoA, support to agriculture is classified into three boxes:
      \begin{enumerate}[label=\Roman*)]
        \item \textbf{Amber box}: Policies that directly affect production (e.g. price support). Countries are allowed a maximum of 5\% (10\% for developing countries) of the Aggregate Measure of Support (AMS) to enter the Amber box (\emph{de minimis} - over total agricultural production).
        \item \textbf{Blue box}: Program payments that are based on fixed acreage and yield, fixed number of head of livestock or if they are designed on 85\% or less of base production. These policies are seen as acceptable but it is difficult to tell whether a subsidy program fits into this category and whether programs in the blue box affect production.
        \item \textbf{Green box}: Policies that have little or no effect on production (research programs, domestic food aid, environmental programs, certain crop insurance and income-support programs).
      \end{enumerate}
   \item The US reformulated its support to agriculture in response to the WTO agreement. The 1996 FAIR Act (farm bill) market price support and deficiency payments were replaced by fixed payments (production flexibility contracts). These payments were calculated by multiplying a farm's payment yield, 85\% of farm's contract acreage and the payment rate.
\end{enumerate}
\end{frame}

%%%%%%%%%%%%%%%%%%%%%%%%%%%%%%%%%%%%%%%%%%%%%%%%%%%%%%%%%%%%%%%%%%%%%%%%%%%%%%%%%%

\begin{frame}{WTO and export subsidies}
\begin{enumerate}[label=\textbullet]
    \item Cap and reduction on the value and volume of export subsidies.
\end{enumerate}
\end{frame}

%%%%%%%%%%%%%%%%%%%%%%%%%%%%%%%%%%%%%%%%%%%%%%%%%%%%%%%%%%%%%%%%%%%%%%%%%%%%%%%%%%

\begin{frame}{Dispute settlement}
\begin{enumerate}[label=\textbullet]
    \item The WTO provides for a system to resolve trade disputes.
    \item In case of dispute a panel of experts is formed to hear the arguments of all parties.
    \item The panel makes a decision and if a party is found not to meet its obligations then it is expected to take action to bring its policies into compliance.
    \item For example, Canada and Mexico have challenged US Country Of Origin Labeling (COOL) policy as it applies to livestock claiming that it violates the national treatment clause.
        \begin{enumerate}[label=-]
        \item Canada and Mexico have repeatedly won but the U.S. has not brought its policies into compliance.
        \item The arbitrator at the WTO authorized in December 2015 Canada and Mexico to retaliate by over \$1 billion.
        \item The same month, the US Congress repealed COOL for red meat and avoided retaliation.
        \end{enumerate}
\end{enumerate}
\end{frame}

%%%%%%%%%%%%%%%%%%%%%%%%%%%%%%%%%%%%%%%%%%%%%%%%%%%%%%%%%%%%%%%%%%%%%%%%%%%%%%%%%%

\begin{frame}{WTO and market access}
\begin{enumerate}[label=\textbullet]
    \item Countries agreed on replacing their non-tariff import barriers (e.g. import quotas) by bound tariff and to reduce the level of those tariffs.
    \item The process of converting non-tariff trade barriers into tariff is call \emph{tariffication}.
    \item Import quotas have been converted into Tariff-rate quotas (TRQs).
    \item A TRQ has three parts:
      \begin{enumerate}[label=\Roman*)]
        \item A low tariff rate ($t_0$) that applies to units imported below the quota quantity.
        \item A quota ($\bar{Q}$).
        \item A high tariff rate ($t_1$) that applies to units imported above the quota quantity.
      \end{enumerate}
    \item Often, the high tariff rate $t_1$ is so high that a TRQ acts as a quota. It is in many cases in the order of several hundred percent.
\end{enumerate}
\end{frame}

%%%%%%%%%%%%%%%%%%%%%%%%%%%%%%%%%%%%%%%%%%%%%%%%%%%%%%%%%%%%%%%%%%%%%%%%%%%%%%%%%%

\begin{frame}{Tariff-rate quotas}
\begin{enumerate}[label=\textbullet]
    \item The following figure shows an example of how a TRQ works.
    \item Suppose that a country is small such that the import is fixed and given by $W$.
    \item The price to the importers is given by $W^d = (1+t)W$, where $t$ is the tariff that applies.
    \item Depending on the import demand $D^I$, the tariff that applies is either $t_0$ or $t_1$.
    \item If the quantity imported $Q^I$ is greater than the quota $\bar{Q}$, then the owners of the import quota earn a rent (i.e. profit).
\end{enumerate}
\end{frame}

%%%%%%%%%%%%%%%%%%%%%%%%%%%%%%%%%%%%%%%%%%%%%%%%%%%%%%%%%%%%%%%%%%%%%%%%%%%%%%%%%%%%%

\begin{frame}{Tariff-rate quota (with import tariff in percentage)}
\begin{figure}[htbp]
\begin{center}
    \begin{picture}(240,180)
        %Axises and labels
        \scriptsize
        \put(0,0){\vector(1,0){240}} %x-axis
        \put(0,0){\vector(0,1){180}} %y-axis
        \put(225,-10){Imports}
        \put(-25,170){Price}
        %World price
        \multiput(0,20)(5,0){48}{\line(1,0){2.5}}%Dashed line
        \put(-10,17){$W$}
        %Tariff schedule
        \multiput(100,0)(0,5){36}{\line(0,1){2.5}}%Dashed line
        \put(95,-10){$\bar{Q}$}
        \thicklines
        %in quota
        \put(0,40){\line(1,0){100}}
        %Quota
        \put(100,40){\line(0,1){80}}
        %Over quota
        \put(100,120){\line(1,0){140}}
        \pause
        %Demand curve
        \onslide<2>
        \put(0,80){\line(1,-1){80}}
        \put(80,10){$D$}
        \put(-40,40){$(1+t_0)W$}
        \thinlines
        \multiput(40,0)(0,5){8}{\line(0,1){2.5}}%Dashed line
        \put(40,-10){$Q^I$}
        \scriptsize
        \put(5,28){\parbox[c]{1in}{Tariff\\revenue}}
        \onslide<3>
        \thicklines
        \multiput(0,60)(5,0){20}{\line(1,0){2.5}}%Dashed line
        \put(0,160){\line(1,-1){160}}
        \put(-15,60){$W^d$}
        \put(160,10){$D$}
        \put(40,48){Rent}
        \put(25,28){Tariff revenue}
        \onslide<4->
        \multiput(0,120)(5,0){48}{\line(1,0){2.5}}%Dashed line
        \put(120,170){\line(1,-1){120}}
        \put(240,60){$D$}
        \put(-40,120){$(1+t_1)W$}
        \put(40,75){Rent}
        \thinlines
        \multiput(170,0)(0,5){24}{\line(0,1){2.5}}%Dashed line
        \put(170,-10){$Q^I$}
        \put(25,28){Tariff revenue}
        \put(110,75){Tariff revenue}
    \end{picture}
\vspace{0.1in}
\caption{Market equilibrium} \label{fig.eq}
\end{center}
\end{figure}
\end{frame}

%%%%%%%%%%%%%%%%%%%%%%%%%%%%%%%%%%%%%%%%%%%%%%%%%%%%%%%%%%%%%%%%%%%%%%%%%%%%%%%%%%

\begin{frame}{Sanitary and Phytosanitary agreement}
\begin{enumerate}[label=\textbullet]
    \item The \href{http://www.wto.org/english/tratop_e/sps_e/spsagr_e.htm}{Sanitary and Phytosanitary} (SPS) agreement prevents countries from adopting and enforcing arbitrary technical measures that limit trade.
    \item Countries have the right to impose sanitary and phytosanitary measures necessary for the protection of human, animal or plant life or health.
    \item However, those measures cannot discriminate between countries.
    \item A country cannot impose a standard on imports that is stricter than what it imposes on domestic products.
    \item Sanitary and phytosanitary measures must have a scientific basis.
    \item One controversy is the use of the precautionary principle by some countries (see \href{http://bch.cbd.int/protocol/}{Cartagena Protocol on biosafety}).
\end{enumerate}
\end{frame}

%%%%%%%%%%%%%%%%%%%%%%%%%%%%%%%%%%%%%%%%%%%%%%%%%%%%%%%%%%%%%%%%%%%%%%%%%%%%%%%%%%

\begin{frame}{Sanitary and Phytosanitary agreement}
\begin{enumerate}[label=\textbullet]
    \item One of the most famous trade dispute is regarding the use of hormones in cattle farming.
    \item In 1988, the EU banned the imports of meat from Canada and the United States because of the use of hormones in cattle farming in those countries.
    \item A \href{http://www.wto.org/english/tratop_e/dispu_e/cases_e/ds26_e.htm}{long dispute} began at the WTO in 1996 regarding the legality of that ban.
    \item Canada and the United States ``won'' and then imposed retaliatory measures against the EU.
    \item An \href{http://www.reuters.com/article/2012/03/14/eu-trade-beef-idUSL5E8EE50620120314}{agreement} was reached in 2012.
\end{enumerate}
\end{frame}

%%%%%%%%%%%%%%%%%%%%%%%%%%%%%%%%%%%%%%%%%%%%%%%%%%%%%%%%%%%%%%%%%%%%%%%%%%%%%%%%%%

\begin{frame}{Doha Development Round of negotiations}
\begin{enumerate}[label=\textbullet]
    \item WTO members began negotiation in the \href{http://www.wto.org/english/tratop_e/dda_e/dda_e.htm}{Doha Development Round} in 2001.
    \item Negotiations are still ``ongoing'', but progress is very slow.
    \item The objectives of the Doha Round were to further liberalize trade with a particular emphasis on measures that favor the development of least-developed nations.
    \item Agriculture has been the major road block in the negotiations.
\end{enumerate}
\end{frame}

%%%%%%%%%%%%%%%%%%%%%%%%%%%%%%%%%%%%%%%%%%%%%%%%%%%%%%%%%%%%%%%%%%%%%%%%%%%%%%%%%
\section{Regional trade agreements}

\begin{frame}{Regional trade agreements}
\begin{enumerate}[label=\textbullet]
    \item Countries can enter into agreements to further facilitate trade. See regional trade agreements on the WTO \href{http://www.wto.org/english/tratop_e/region_e/region_e.htm}{website}.
    \item These agreements are also referred to as \emph{preferential trade agreements}.
    \item For example, the United States, Canada and Mexico are members of the North American Free Trade Agreement (NAFTA).
    \item The US is part of many other regional trade agreements. See the WTO \href{http://www.wto.org/english/tratop_e/region_e/rta_participation_map_e.htm}{database}.
    \item The Trans-Pacific Partnership (\href{https://ustr.gov/tpp/}{TPP}) is currently a very much debated trade agreement.
\end{enumerate}
\end{frame}

%%%%%%%%%%%%%%%%%%%%%%%%%%%%%%%%%%%%%%%%%%%%%%%%%%%%%%%%%%%%%%%%%%%%%%%%%%%%%%%%%
\section{Exchange rate}

\begin{frame}{Exchange rate}
\begin{enumerate}[label=\textbullet]
    \item The exchange rate matters when trading commodities.
    \item For example, if \$US 1.00 = \$CA 1.35, it means that it costs 1.35 Canadian dollar to buy one US dollar.
    \item Variations in exchange rate will affect trade volumes.
    \item Possible to buy futures for the exchange rate and hence hedge.
\end{enumerate}
\end{frame}
%
%%%%%%%%%%%%%%%%%%%%%%%%%%%%%%%%%%%%%%%%%%%%%%%%%%%%%%%%%%%%%%%%%%%%%%%%%%%%%%%%%%
\begin{frame}{Canada-US exchange rate}
\begin{knitrout}
\definecolor{shadecolor}{rgb}{0.969, 0.969, 0.969}\color{fgcolor}\begin{kframe}


{\ttfamily\noindent\bfseries\color{errorcolor}{\#\# Error in print(CAexchangerate): object 'CAexchangerate' not found}}\end{kframe}
\end{knitrout}
\scriptsize
Source: The data are from \href{http://www.federalreserve.gov/econresdata/default.htm}{Federal Reserve}, downloaded from \href{http://www.quandl.com/}{Quandl}.
\end{frame}

%%%%%%%%%%%%%%%%%%%%%%%%%%%%%%%%%%%%%%%%%%%%%%%%%%%%%%%%%%%%%%%%%%%%%%%%%%%%%%%%%%
\begin{frame}{Euro-US exchange rate}
\begin{knitrout}
\definecolor{shadecolor}{rgb}{0.969, 0.969, 0.969}\color{fgcolor}\begin{kframe}


{\ttfamily\noindent\bfseries\color{errorcolor}{\#\# Error in print(EUexchangerate): object 'EUexchangerate' not found}}\end{kframe}
\end{knitrout}
\scriptsize
Source: The data are from \href{http://www.federalreserve.gov/econresdata/default.htm}{Federal Reserve}, downloaded from \href{http://www.quandl.com/}{Quandl}.
\end{frame}

%%%%%%%%%%%%%%%%%%%%%%%%%%%%%%%%%%%%%%%%%%%%%%%%%%%%%%%%%%%%%%%%%%%%%%%%%%%%%%%%%%
\begin{frame}{China-US exchange rate}
\begin{knitrout}
\definecolor{shadecolor}{rgb}{0.969, 0.969, 0.969}\color{fgcolor}\begin{kframe}


{\ttfamily\noindent\bfseries\color{errorcolor}{\#\# Error in print(Chinaexchangerate): object 'Chinaexchangerate' not found}}\end{kframe}
\end{knitrout}
\scriptsize
Source: The data are from \href{http://www.federalreserve.gov/econresdata/default.htm}{Federal Reserve}, downloaded from \href{http://www.quandl.com/}{Quandl}.
\end{frame}


%%%%%%%%%%%%%%%%%%%%%%%%%%%%%%%%%%%%%%%%%%%%%%%%%%%%%%%%%%%%%%%%%%%%%%%%%%%%%%%%%

\begin{frame}{What determines an exchange rate}
\begin{enumerate}[label=\textbullet]
    \item The exchange rate is determined by market forces.
    \item The intersection of the demand and supply for a currency determines its exchange rate relative to other currency.
    \item For example, Canada is an oil exporter. If the price of oil increases, the demand for \$CA increases because oil buyers must use \$CA to purchase Canada oil causing the value of \$CA to increase relative to \$US.
\end{enumerate}
\end{frame}

%%%%%%%%%%%%%%%%%%%%%%%%%%%%%%%%%%%%%%%%%%%%%%%%%%%%%%%%%%%%%%%%%%%%%%%%%%%%%%%%%%

\begin{frame}{Canada-US exchange rate}
\begin{figure}[htbp]
\begin{center}
    \begin{picture}(240,180)
        %Axises and labels
        \scriptsize
        \put(0,0){\vector(1,0){240}} %x-axis
        \put(0,0){\vector(0,1){180}} %y-axis
        \put(225,-10){\$CA}
        \put(-38,170){\$CA/\$US}
        %Demand curve
        \thicklines
        \put(0,140){\line(1,-1){140}}
        %Supply curve
        \put(0,0){\line(1,1){160}}
        %Text
        \put(135,10){$D$}
        \put(150,155){$S$}
        %Equilibrium
        \color{black}
        \multiput(0,70)(5,0){14}{\line(1,0){2.5}}%Dashed line
        \multiput(70,70)(0,-5){14}{\line(0,-1){2.5}}%Dashed line
        \put(-10,68){$e^\ast$}
        \put(68,-10){$Q^\ast$}
        %Positive shift in demand
        \onslide<2>
        \color{blue}
        \put(0,160){\line(1,-1){160}}
        \put(40,100){\vector(1,0){20}} %x-axis
        \put(155,10){$D_0$}
        \multiput(0,80)(5,0){16}{\line(1,0){2.5}}%Dashed line h
        \multiput(80,80)(0,-5){16}{\line(0,-1){2.5}}%Dashed line v
        \put(78,-10){$Q_0$}
        \put(-10,78){$e_0$}
        %Negative shift in demand
        \onslide<3>
        \color{red}
        \put(0,120){\line(1,-1){120}}
        \put(40,100){\vector(-1,0){20}} %x-axis
        \put(115,10){$D_1$}
        \multiput(0,60)(5,0){12}{\line(1,0){2.5}}%Dashed line h
        \multiput(60,60)(0,-5){12}{\line(0,-1){2.5}}%Dashed line v
        \put(58,-10){$Q_1$}
        \put(-10,58){$e_1$}
    \end{picture}
\end{center}
\end{figure}
\end{frame}

%%%%%%%%%%%%%%%%%%%%%%%%%%%%%%%%%%%%%%%%%%%%%%%%%%%%%%%%%%%%%%%%%%%%%%%%%%%%%%%%%

\begin{frame}{Who trades currencies?}
\begin{enumerate}[label=\textbullet]
    \item Those that demand US dollars are:
      \begin{enumerate}[label=-]
        \item Buyers of US commodities (e.g. firms that import goods from the United States);
        \item Tourists;
        \item Speculators, hedgers.
      \end{enumerate}
    \item Those that supply US dollars are:
      \begin{enumerate}[label=-]
        \item US importers;
        \item US government;
        \item US tourists visiting other countries;
        \item Speculators, hedgers.
      \end{enumerate}
\end{enumerate}
\end{frame}

%%%%%%%%%%%%%%%%%%%%%%%%%%%%%%%%%%%%%%%%%%%%%%%%%%%%%%%%%%%%%%%%%%%%%%%%%%%%%%%%%

\begin{frame}{Currency value and trade}
\begin{enumerate}[label=\textbullet]
    \item Changes in currency values affect relative prices across countries.
    \item If the value of Canadian dollar falls, that means it becomes relatively cheaper for Americans to import products from Canada.
    \item This means that the depreciation of a country's currency can boost its exports.
    \item This is why China has devaluated its currency lately.
\end{enumerate}
\end{frame}

%%%%%%%%%%%%%%%%%%%%%%%%%%%%%%%%%%%%%%%%%%%%%%%%%%%%%%%%%%%%%%%%%%%%%%%%%%%%%%%%%

\begin{frame}{Currency value and trade}
\begin{enumerate}[label=\textbullet]
    \item Most exchange rates are free to move.
    \item A few countries have been manipulating their currency.
    \item China has been manipulating the Renminbi for quite some time in an effort to favor its exports.
    \item China has let its currency move more freely in recent years.
    \item Many countries oppose China's currency policy and this may bring WTO complaints, but there is no apparent role for the WTO to play in this issue. IMF?
\end{enumerate}
\end{frame}


%%%%%%%%%%%%%%%%%%%%%%%%%%%%%%%%%%%%%%%%%%%%%%%%%%%%%%%%%%%%%%%%%%%%%%%%%%%%%%%%%%

\begin{frame}{Exchange rate and agricultural markets}
\begin{enumerate}[label=\textbullet]
    \item Of course the exchange rate impacts agricultural markets.
    \item The demand for US agricultural commodities increases if the US dollar depreciates.
    \item The demand for US agricultural commodities decreases if the US dollar appreciates.
    \item US will import more from country which currency depreciates.
    \item US will import less from country which currency appreciates.
\end{enumerate}
\end{frame}

%%%%%%%%%%%%%%%%%%%%%%%%%%%%%%%%%%%%%%%%%%%%%%%%%%%%%%%%%%%%%%%%%%%%%%%%%%%%%%%%%%

\begin{frame}{Exchange rate and agricultural markets}
\begin{enumerate}[label=\textbullet]
    \item Changes in a currency values usually respond to countries relative economic strength.
    \item As agriculture is a small share of most economies, agriculture has a marginal effect on exchange rates.
    \item Appreciation of a country's currency may boost its demand for a US commodity, thus increasing the price of that commodity in the United States.
\end{enumerate}
\end{frame}

%%%%%%%%%%%%%%%%%%%%%%%%%%%%%%%%%%%%%%%%%%%%%%%%%%%%%%%%%%%%%%%%%%%%%%%%%%%%%%%%%%

\section{Trends in global agricultural markets}

\begin{frame}{Trends in global agricultural markets}
\begin{enumerate}[label=\textbullet]
    \item Agricultural prices were relatively stable between 1990 and 2007.
    \item Then, in 2007, the price of food practically doubled and became more volatile.
    \item Several competing explanations of the price boom:
      \begin{enumerate}[label=-]
        \item Speculation (not credible);
        \item Ethanol policy in the United States (may have contributed);
        \item Contemporaneous supply and demand surprises that coincide with low inventories and macroeconomic shocks (see \href{http://www.annualreviews.org/doi/abs/10.1146/annurev.resource.012809.104220?journalCode=resource}{Carter, Rausser and Smith, 2011}).
      \end{enumerate}
    \item Note that there was also a price boom of agricultural commodity prices in the 1970s.
\end{enumerate}
\end{frame}


%%%%%%%%%%%%%%%%%%%%%%%%%%%%%%%%%%%%%%%%%%%%%%%%%%%%%%%%%%%%%%%%%%%%%%%%%%%%%%%%%%
\begin{frame}{FAO World Food Price Index}
\begin{knitrout}
\definecolor{shadecolor}{rgb}{0.969, 0.969, 0.969}\color{fgcolor}\begin{kframe}


{\ttfamily\noindent\bfseries\color{errorcolor}{\#\# Error in print(WorldFoodindex): object 'WorldFoodindex' not found}}\end{kframe}
\end{knitrout}
\scriptsize
Source: The data are from \href{http://www.fao.org/worldfoodsituation/foodpricesindex/en/}{FAO}. Average nominal price index equals 100 for the period of 2002-2004.
\end{frame}

%%%%%%%%%%%%%%%%%%%%%%%%%%%%%%%%%%%%%%%%%%%%%%%%%%%%%%%%%%%%%%%%%%%%%%%%%%%%%%%%%%
\begin{frame}{FAO World Cereals Price Index}
\begin{knitrout}
\definecolor{shadecolor}{rgb}{0.969, 0.969, 0.969}\color{fgcolor}\begin{kframe}


{\ttfamily\noindent\bfseries\color{errorcolor}{\#\# Error in print(WorldCerealsindex): object 'WorldCerealsindex' not found}}\end{kframe}
\end{knitrout}
\scriptsize
Source: The data are from \href{http://www.fao.org/worldfoodsituation/wfs-home/foodpricesindex/en/}{FAO}. Average nominal price index equals 100 for the period of 2002-2004.
\end{frame}

%%%%%%%%%%%%%%%%%%%%%%%%%%%%%%%%%%%%%%%%%%%%%%%%%%%%%%%%%%%%%%%%%%%%%%%%%%%%%%%%%%
\begin{frame}{FAO World Oils Price Index}
\begin{knitrout}
\definecolor{shadecolor}{rgb}{0.969, 0.969, 0.969}\color{fgcolor}\begin{kframe}


{\ttfamily\noindent\bfseries\color{errorcolor}{\#\# Error in print(World\_Oils\_index): object 'World\_Oils\_index' not found}}\end{kframe}
\end{knitrout}
\scriptsize
Source: The data are from \href{http://www.fao.org/worldfoodsituation/wfs-home/foodpricesindex/en/}{FAO}. Average nominal price index equals 100 for the period of 2002-2004.
\end{frame}

%%%%%%%%%%%%%%%%%%%%%%%%%%%%%%%%%%%%%%%%%%%%%%%%%%%%%%%%%%%%%%%%%%%%%%%%%%%%%%%%%%
\begin{frame}{FAO World Dairy Price Index}
\begin{knitrout}
\definecolor{shadecolor}{rgb}{0.969, 0.969, 0.969}\color{fgcolor}\begin{kframe}


{\ttfamily\noindent\bfseries\color{errorcolor}{\#\# Error in print(World\_Dairy\_index): object 'World\_Dairy\_index' not found}}\end{kframe}
\end{knitrout}
\scriptsize
Source: The data are from \href{http://www.fao.org/worldfoodsituation/wfs-home/foodpricesindex/en/}{FAO}. Average nominal price index equals 100 for the period of 2002-2004.
\end{frame}

%%%%%%%%%%%%%%%%%%%%%%%%%%%%%%%%%%%%%%%%%%%%%%%%%%%%%%%%%%%%%%%%%%%%%%%%%%%%%%%%%%
\begin{frame}{FAO World Meat Price Index}
\begin{knitrout}
\definecolor{shadecolor}{rgb}{0.969, 0.969, 0.969}\color{fgcolor}\begin{kframe}


{\ttfamily\noindent\bfseries\color{errorcolor}{\#\# Error in print(WorldMeatindex): object 'WorldMeatindex' not found}}\end{kframe}
\end{knitrout}
\scriptsize
Source: The data are from \href{http://www.fao.org/worldfoodsituation/wfs-home/foodpricesindex/en/}{FAO}. Average nominal price index equals 100 for the period of 2002-2004.
\end{frame}

%%%%%%%%%%%%%%%%%%%%%%%%%%%%%%%%%%%%%%%%%%%%%%%%%%%%%%%%%%%%%%%%%%%%%%%%%%%%%%%%%%

\section{Conclusions}

\begin{frame}{Conclusions}
\begin{enumerate}[label=\textbullet]
    \item International markets are becoming more and more important;
    \item Increase in income, especially in developing countries, drives new demands for agricultural commodities, especially for luxury food products (e.g. meat);
    \item The eventual conclusion of the Doha Round of negotiations and the signing of regional trade agreements should further contribute to free trade for agricultural commodities.
\end{enumerate}
\end{frame}


%%%%%%%%%%%%%%%%%%%%%%%%%%%%%%%%%%%%%%%%%%%%%%%%%%%%%%%%%%%%%%%%%%%%%%%%%%%%%%%%%%%%%
%\section[References]{References}
%\renewcommand\refname{References}
%\def\newblock{References}
%\begin{frame}[allowframebreaks]{References}
%\bibliography{R:/users/pouliot/Papers/References}
%%\bibliography{C:/Users/pouliot/Documents/Papers/References}
%\end{frame}


%%%%%%%%%%%%%%%%%%%%%%%%%%%%%%%%%%%%%%%%%%%%%%%%%%%%%%%%%%%%%%%%%%%%%%%%%%%%%%%%%%%%%

\end{document}

